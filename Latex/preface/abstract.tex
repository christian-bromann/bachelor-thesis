\vspace*{2cm}

\begin{center}
    \textbf{Abstract}
\end{center}

\vspace*{1cm}

\noindent Das Internet hat sich zu einer der wichtigsten Werkzeuge zur Kommunikation des Menschen entwickelt und nimmt eine immer wichtigere Rolle in unserem Alltag ein. Umso wichtiger ist es, dass der Mensch dieses Werkzeug richtig zu benutzen weiß, um dessen volle Funktionalität effizient ausschöpfen zu können. Um dies zu Erreichen wurde im Zuge dieser Arbeit ein Usability-Testing-Tool mit dem Namen \textit{thEvaluator} entwickelt, mithilfe dessen das Nutzerverhalten auf beliebigen Webseiten untersucht werden kann, um daraus Probleme zu evaluieren und Lösungsmöglichkeiten zu entwickeln. Nach der Durchführung zweier Tests mit dem Programm, konnten einmal am Tool selbst und an einer Hochschulseite insgesamt fünf wichtige Erkenntnisse über die Benutzung ausgewertet werden. Durch den Einsatz von verschiedensten Techniken und Metriken ermöglicht der entwickelte Prototyp eine flexible Untersuchung der Usability und kann dadurch bei der Erstellung von benutzerfreundlichen Internetseiten helfen.
