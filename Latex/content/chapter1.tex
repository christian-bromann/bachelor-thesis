\chapter*{Abstract}%
\addcontentsline{toc}{chapter}{\numberline{}Abstract}%

\chapter{Einleitung}
	\section{Motivation und Ausgangssituation}
	\section{Projektziel}
	\section{Aufbau der Arbeit}

\chapter{Usability}
% Was ist Usability
% http://www.webpagecontent.com/arc_archive/124/5/
	\section{Usability im Web}
	% Usability l�sst sich nicht via Testscript Testen, sondern h�ngt immer ab von den verschiedenen Einfl�ssen der Benutzer
	% http://usabilitygeek.com/an-introduction-to-website-usability-testing/
	% http://usabilitygeek.com/wp-content/uploads/2011/06/International-Journal-of-Human-Computer-Interaction-IJHCI-Volume-2-Issue-1.pdf
	\section{Einfl�sse}
	% Was kann die Benutzbarkeit beeinflussen
	% Barrierefreiheit
	% X-Browser Gleichheit
	\section{Metriken}
		\subsection{Clickmaps}
		\subsection{Heatmaps}
		\subsection{Gazespots}
		\subsection{Gazeplots}
	\section{Marktanalyse}
	% Analyse schon vorhandener Produkte
	% zbs: loop11.com, met.picnet.com.au, crazyegg.com
	\section{Mensch vs. Maschiene}
	% in wie weit kann durch Testscripts die Usability getestet werden
	% webdriverjs
	\section{Business-Case}
	% m�gliche Testprozesse
	% automatische vs. gesteuert

\chapter{thEvaluator}
	\section{Zielbestimmungen}
	% Entwickler muss f�r bestimmte Sachen Events definieren
	% vs.
	% Tool kommt ohne Definierung von Events aus
	\section{Produktfunktionen}
	% Was ist m�glich, wie weit gehen die Auswertungen
	\section{Produkteinsatz}
	% Wie/Wann kann das Tool genutzt werden
	\section{Extension}
		\subsection{Aufbau}
		% Aus was besteht eine Chrome Extension
		% Manifest erkl�ren
		% Einbindung ohne Google Store
		\subsection{Architektur}
		% http://developer.chrome.com/extensions/overview.html
		\subsection{Nachrichten�bermittlung}
		% http://developer.chrome.com/extensions/messaging.html
		\subsection{Datenaufzeichnung}
		% Welche Daten werden wann aufgezeichnet
		% Screenshot Erstellung
		% Problematik der erh�hten Datenmenge
	\section{API}
		\subsection{Architektur}
		% express, welche Models gibt es
		% Welche Dependencies , Deploymentprozess
		\subsection{Asynchronit�t in nodeJS}
		% M�glichkeiten zur L�sung �ber closures, callbacks, Modulen wie async
		\subsection{REST Schnittstellen}
		% Auflistung mit erforderten Werten / R�ckgabewerten
		\subsection{Socket Streams}
		% Auflistung mit erforderten Werten / R�ckgabewerten
		\subsection{Persistierung}
		% SQL vs NoSQL
		% MongoDB / Mongoose
	\section{Webapplikation}
		\subsection{Aufbau}
		% Tooling, Bower, NPM
		% State of the art frontend entwicklung
		% Welche Dependencies , Deploymentprozess
		\subsection{Architektur}
		% Backbone, Models (die DB Modelle abbilden) Collections, Views
		% MV* Erkl�rung
		\subsection{TestCase Definition}
		% warum welche Attribute und Bedeutung (zbs. Aufl�sung, required)
		\subsection{Datenaufbereitung}
		% wann werden welche Daten geladen (lazy loading)
		% handling der Datenmenge
		\subsection{Datenevaluation}
		% welche Auswertungswidgets gibt es / Bedeutung
		% wann kann man m�gliche Resultate ableiten
		\subsection{Qualit�tsanforderungen}
	\section{Zukunftsaussichten}
	% Einsatz der Webcam f�r Eyetracking
	
\chapter{Ergebnis}
	\section{Scope-Verschiebung w�hrend der Entwicklung}
	% Heatmaps sind nicht mehr das Einzige, was es wert ist zu untersuchen
	% thEvaluator als universelles Usability Tool
	\section{Evaluierung der Hochschulseite}
	\section{Ergebnisse}
	\section{Schl�sse}

