%
% Zukunftsaussichten
% Abschlussarbeit (Bachelor)
%
% Thema: Erstellung einer Browser Extension zur Usability Evaluierung von beliebigen Web-Applikationen über Heatmaps.
% Betreuer 1: Prof. Dr. Targo Pavlista
% Betreuer 2: Siamak Haschemi
%
% @author Christian Bromann <contact@christian-bromann.com>
%

\section{Ausblick}

Wie schon oft in der Arbeit erwähnt, befindet sich \textit{thEvaluator} in einem Prototyp Status. Bei der Nutzung fallen dem User noch vereinzelt Fehler auf, die es zu beseitigen gilt. Ganz Abseits dieser Probleme bietet die Applikation allerdings interessante Möglichkeiten zur Weiterentwicklung an. Es könnten zum Einen weitere Browser-APIs, wie z.B. die \textit{MediaStream Processing API}\footnote{\url{https://developer.mozilla.org/en-US/docs/WebRTC/MediaStream_API}}, eingesetzt werden. Durch die Nutzung des im Computer eingebauten Mikrofons kann dadurch der User während des Testlaufes seine Eindrücke mündlich äußern. Die Extension würde dies aufnehmen und verarbeiten. Zur Auswertung könnte sich dann der Evaluator die Aufnahme anhören und daraus wichtige Erkenntnisse schließen. In Verbindung mit der \textit{Web Speech API}\footnote{\url{http://updates.html5rocks.com/2013/01/Voice-Driven-Web-Apps-Introduction-to-the-Web-Speech-API}} lässt sich dieses Feature noch weiter entwickeln. Die API erzeugt aus dem gesprochenen Wort ein Text, mit dem eine bessere Auswertung des Feedbacks geboten wäre.\\
\\
Des Weiteren ist die Einbeziehung der Kamera ein mögliches Hilfsmittel, um mehr über das Nutzungsverhalten der User herauszubekommen. Hier interessiert vor allem die Bewegung des Auges. Die Extension könnte die Aufgabe eines Eye-Tracking-Programmes übernehmen, um daraus echte Heatmaps zu erzeugen. In der aktuellen Version dient die Maus als Ersatz für das Auge. Da viele Menschen die Maus benutzen, um sich auf einer Seite zu orientieren, bietet dieser Ansatz eine verlässliche Auswertung. Nutzt die Applikation jedoch die Möglichkeit des Eye-Trackings, würden die Daten viel genauere und realistischere Ergebnisse liefern. Der Vergleich zwischen Maus und Augenposition, z.B. durch Messung der Entfernung beider Punkte, könnte bspw. sehr interessante Rückschlüsse darüber geben, wie Erfahren der Benutzer im Umgang mit Computern ist.\\
\\
Neben der Nutzung weiterer Technologien ist die Zugangs-Erweiterung des Frameworks ebenfalls ein möglicher Punkt, für den Ausbau der Applikation. Aktuell benötigt ein Evaluator einen Testcase und Probanden, um Informationen über die Usability der eigenen oder einer fremden Seite zu erhalten. Oft besteht dabei jedoch nicht die Anforderung, einen Testcase für die Evaluierung der Seite zu benutzen. Ein Script, welches in die Seite eingebunden wird, könnte die Aufgaben der Extension übernehmen und die Daten der täglichen Besucher aufzeichnen. Der initiale Aufwand dafür ist sehr gering und kann in wenigen Minuten erledigt sein. Die aufgezeichneten Daten sagen dann zwar nichts über das Vorhaben der Nutzer aus, liefern dennoch viele Informationen, die für die Verbesserung der Usability genutzt werden können.