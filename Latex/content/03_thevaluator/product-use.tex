%
% Produkteinsatz
% Abschlussarbeit (Bachelor)
%
% Thema: Erstellung einer Browser Extension zur Usability Evaluierung von beliebigen Web-Applikationen über Heatmaps.
% Betreuer 1: Prof. Dr. Targo Pavlista
% Betreuer 2: Siamak Haschemi
%
% @author Christian Bromann <contact@christian-bromann.com>
%

\section{Produkteinsatz}

Für eine erfolgreiche Evaluierung der Usability einer Webseite mit \textit{thEvaluator} werden als erstes Probanden benötigt, die bereit sind, sich die Extension zu installieren und den Test durchzuführen. Die Anzahl der Teilnehmer kann zwischen 5 und 15 Personen liegen. Als nächstes muss der Testcase und die darin enthaltenen Aufgaben definiert werden. Diese sollten klug gewählt sein und den Proband auf mögliche Usability-Problemstellen leiten. Wichtig ist auch, dass die Fragestellung klar und verständlich formuliert wird. Der User muss beim Test sofort wissen, was er zu tun hat. Zusätzlich ist es möglich, durch die Fragen eine Art Geschichte zu erzählen, die den Proband dazu bringt, sich besser in eine spezifische Situation hineinzuversetzen.\\
\\
Die Geschichte kann dabei Situationen beinhalten, deren Ausgang die Fortsetzung bestimmt. Scheitert der User an einer Aufgabe, so kann es vorkommen, dass die Folgeaufgaben nicht ausführbar sind. Es ist deshalb möglich, sie als \glqq erforderlich\grqq{} zu markieren. Dadurch wird der Test beendet, wenn der User es nicht schafft, die Aufgabe zu lösen. Der Testlauf wird dann dementsprechend auch in der Auswertung als \glqq durchgefallen\grqq{} gekennzeichnet. Bei einem Usability-Test eines Shop-Systems kann der User bspw. keine Bestellung durchführen, wenn er es nicht einmal schafft, einen Artikel in den digitalen Warenkorb zu legen.\\
\\
Eine weitere Komponente der Aufgabe ist die Zeit. Der Benutzer sollte nicht zu lange brauchen, einen Task zu erfüllen. In einer realen Situation würde dieser die Seite irgendwann verlassen, da er ungeduldig wird und sein Glück auf einer anderen Webseite sucht. Die Zeit ist deshalb ein Faktor dafür, ob die Aufgabe bestanden ist oder nicht. Läuft sie ab, so endet der Task und der nächste startet automatisch. Ist die Aufgabe jedoch als \glqq erforderlich\grqq{} markiert, so endet der Test.\\
\\
Nachdem die Aufgaben zusammengestellt und der Testcase erzeugt wurde, bekommt diese eine 10-stellige ID zugewiesen. Dies ist der Schlüssel zum Starten des Tests. Er sollte mit ein paar Erläuterungen an die Testuser geschickt werden. Diese geben den Code in die Extension ein und beginnen damit den Test. Es öffnet sich ein neuer Tab mit der im Testcase definierten Seite. Sobald diese geladen ist wird die erste Frage angezeigt. Fortan zeichnet die Extension jegliche Bewegungen der Maus und dessen Klicks auf. Der Test ist beendet, sobald der Benutzer alle Aufgaben erfolgreich gelöst hat oder die Zeit für einen erforderlichen Task abgelaufen ist. Er hat danach noch die Möglichkeit, in einem Textfeld individuelles Feedback zu geben und seine Erfahrungen, die er während des Tests gemacht hat, aufzuschreiben.