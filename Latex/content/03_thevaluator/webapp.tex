%
% Webapplikation
% Abschlussarbeit (Bachelor)
%
% Thema: Erstellung einer Browser Extension zur Usability Evaluierung von beliebigen Web-Applikationen über Heatmaps.
% Betreuer 1: Prof. Dr. Targo Pavlista
% Betreuer 2: Siamak Haschemi
%
% @author Christian Bromann <contact@christian-bromann.com>
%

\section{Webapplikation}

Die \textbf{API} ist der Datenmotor des Frameworks. Sie ist sowohl für die Datenpersistierung als auch für dessen Auslieferung zuständig.

% Was ist mšglich, wie weit gehen die Auswertungen
% Walkpath map: cognitives model: "journal of emerging technologies"

\subsection{Aufbau}
% Tooling, Bower, NPM
% State of the art frontend entwicklung
% Welche Dependencies , Deploymentprozess

\subsection{Architektur}
% Backbone, Models (die DB Modelle abbilden) Collections, Views
% MV* ErklŠärung

\subsection{TestCase Definition}
% warum welche Attribute und Bedeutung (zbs. Auflšösung, required)

\subsection{Datenaufbereitung}
% wann werden welche Daten geladen (lazy loading)
% handling der Datenmenge

\subsection{Datenevaluation}
% welche Auswertungswidgets gibt es / Bedeutung
% bei der Erklärung des walkpath widgets, dies mit einbauen
% Die Analyse von Verhaltens- und Navigationsmustern und der anschließenden Verbesserung der Link Struktur einer Seite fällt unter den Begriff des \textit{Web-Usage-Mining}, welches ein Untersuchungsgegenstand des \textit{Web Minings} ist. \cite{webusagemining}. Auf Grundlage dieser Navigationsmuster lassen sich Pfade herleiten und Linkstrukturen berechnen. Wissenschaftler aus China haben daraus eine neuartige Methode entwickelt, die Qualität der Linkstruktur mathematisch zu berechnen \cite{linkStructure}.\\
% \\
% Es wird davon ausgegangen, dass die Link Struktur einer Seite durch \textbf{G} in \ref{structure} als gewichtet-direkterer Graph repräsentiert wird.
% 
% \begin{equation}
%     G = (N, P, L, W)
%     \label{structure}
% \end{equation}
% 
% \begin{itemize}
%     \item N =  Anzahl der Seiten einer Website (Anzahl der Knoten des Graphen)
%     \item P = Menge aller Knoten in G, $ \{P_i | i \in [1,N]\} $
%     \item L = Menge aller Kanten in G, $ \{L_{i,j} | i \neq j; i,j \in [1,N]\} $\\
%     	     ($ L_{i,j} $ entspricht dabei ein Link von $ P_i $ zu $ P_j $)
%     \item W = Menge aller Knoten-Gewichtungen in G, $\{W_{i,j} | i \neq j; i,j \in [1,N]\}$\\
%              \\
%              Die Wahrscheinlichkeit dafür, das der Besucher auf der Seite $P_i$ einen % Link $L_{i,j}$ folgt und dadurch auf $P_j$ gelangt wird
%              durch $W_{i,j}$ gekennzeichnet und berechnet sich wie folgt:\\
%              \\
%              $W_{i,j} = \frac{V_{i,j}}{ \sum_{k = 1}^{D_i} V_{idk}}$\\
%              \\
%              $D_i$ ist definiert als Menge aller Seiten, die auf $P_i$ durch Links erreichbar sind.\\
%              $V_{i,j}$ wird dadurch zum Bogenmaß der Kante von $P_i$ zu $P_j$
% \end{itemize}

% wann kann man mšögliche Resultate ableiten

\subsection{Qualitätsanforderung}