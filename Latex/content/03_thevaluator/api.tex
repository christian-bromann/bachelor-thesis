%
% API
% Abschlussarbeit (Bachelor)
%
% Thema: Erstellung einer Browser Extension zur Usability Evaluierung von beliebigen Web-Applikationen über Heatmaps.
% Betreuer 1: Prof. Dr. Targo Pavlista
% Betreuer 2: Siamak Haschemi
%
% @author Christian Bromann <contact@christian-bromann.com>
%

\section{API}

Die \textbf{API} ist der Datenmotor des Frameworks. Sie ist sowohl für die Datenpersistierung als auch für dessen Auslieferung zuständig.

\subsection{Architektur}
% express, welche Models gibt es
% Welche Dependencies , Deploymentprozess

\subsection{Asynchronität in nodeJS}
% Möšglichkeiten zur Lšösung Ÿüber closures, callbacks, Modulen wie async

\subsection{REST Schnittstellen}
% Auflistung mit erforderten Werten / RŸückgabewerten

\subsection{Socket Streams}
% Auflistung mit erforderten Werten / RüŸckgabewerten

\subsection{Persistierung}
% SQL vs NoSQL
% MongoDB / Mongoose