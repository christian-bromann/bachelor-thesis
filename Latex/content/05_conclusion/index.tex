%
% Ergebnis
% Abschlussarbeit (Bachelor)
%
% Thema: Erstellung einer Browser Extension zur Usability Evaluierung von beliebigen Web-Applikationen über Heatmaps.
% Betreuer 1: Prof. Dr. Targo Pavlista
% Betreuer 2: Siamak Haschemi
%
% @author Christian Bromann <contact@christian-bromann.com>
%

\chapter{Ergebnis}

Das Ziel dieser Arbeit war die Entwicklung eines Usability-Tools zur Analyse des Nutzerverhaltens auf beliebigen Webseiten über Heatmaps. Die im Rahmen des Tools definierten Testcases sollten einen aufgabenbasierten Charakter besitzen, um Verhaltensmuster der Besucher genauer abbilden zu können. Nach der Entwicklung des Tools, sollten mit zwei Tests die Funktionalität des Tools auf die Probe gestellt und die Usability von Hochschulseiten überprüft werden.\\
\\
Mit \textit{thEvaluator} ist ein Tool geschaffen worden, mit dem präzise Usability-Evaluierungen in Echtzeit möglich sind. Die verschiedenen Auswertungs-Widgets bieten eine flexible Analyse des Nutzerverhaltens und die Verfolgung der Mauspositionen eines jeden Benutzers an. Durch den Einsatz einer Browser-Extension zur Aufzeichnung der Daten kann das Tool auf jeder beliebigen Seite eingesetzt werden. Es setzt keine Code-Anpassungen voraus. Obwohl es dabei auf den neuesten Web-Technologien aufbaut, ist die Portierung der Extension auf andere Browser problemlos umsetzbar. Zudem ermöglichen die aufgabenbasierten Testcases die Nachahmung bestimmter Nutzungsszenarien. Der Einsatz von NodeJS als API-Lösung und Socket-Streams zur Datenübermittlung macht das Tool sehr performant und flexibel erweiterbar. Der mit dieser Arbeit entwickelte Prototyp lässt sich durch weitere Browser-APIs ergänzen und stellt eine echte Konkurrenz gegenüber den teuren Produkten auf dem Markt dar.\\
\\
Während der Entwicklung des Tools stellte sich heraus, dass für eine genaue Untersuchung der Usability eine Heatmap nicht ausreicht. Durch die Ergänzung von Clickmaps, Gazespots und Gazeplots, einer Feedbackbox, sowie der zeitlich versetzten Rekonstruktion der Mausbewegung, wurden weitere Möglichkeiten zur Analyse des Userverhaltens ergänzt. Zudem ist durch die Implementierung eines Baum-Diagrammes die Untersuchung von Besucherpfaden möglich. Dadurch hat das Tool den Charakter eines universalen Usability-Analyse-Werkzeuges erhalten, welches sich nicht nur auf Heatmaps beschränkt.\\
\\
Die Usability-Tests am Ende der Arbeit gaben erste Erkenntnisse über die Benutzerfreundlichkeit des Tools selber und zeigten verschiedene Usability-Problemstellen auf der Seite der Beuth Hochschule auf. Die Analyse der Daten ergab für den ersten Test, dass das Formular zur Erstellung eines Testcases für den Benutzer nicht selbsterklärend genug ist und durch Hinweistexte ergänzt werden sollte. Bei der Analyse der Hochschulseite stellte sich heraus, dass es vereinzelt Usability-Probleme gibt und die Probanden beim Finden von Informationen vermehrt die Such-Funktion genutzt haben. Dennoch fiel das Ergebnis des Tests positiv aus und bewies, dass Besucher von Hochschulseiten zwar lange zum Finden von bestimmten Informationen brauchen, jedoch die Suche häufig erfolgreich ausgeht.\\
\\
Das Tool konnte mit seinen Auswertungsmethoden viele Erkenntnisse über das Verhalten der Besucher aufzeigen. Zudem half die späte Implementierung der Feedbackbox bei der Auswertung sehr, verschiedene Erfahrungen der Nutzer einzusammeln. Alles in Allem ist das Tool ein Prototyp und enthält an vereinzelten Stellen noch Bugs, die behoben werden müssen. Durch die Nutzung weiterer Browser-APIs bietet es jedoch auch viele spannende Möglichkeiten zur Weiterentwicklung.