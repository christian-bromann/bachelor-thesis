%
% Projektziel
% Abschlussarbeit (Bachelor)
%
% Thema: Erstellung einer Browser Extension zur Usability Evaluierung von beliebigen Web-Applikationen über Heatmaps.
% Betreuer 1: Prof. Dr. Targo Pavlista
% Betreuer 2: Siamak Haschemi
%
% @author Christian Bromann <contact@christian-bromann.com>
%

\section{Projektziel}

Hochschulseiten haben es schwer die Fülle an Informationen optimal auf einer Website zu platzieren oder zu verlinken. Da die Zielgruppe der Besucher, von Abiturabsolventen bis hin zu Professoren und Alumni sehr breit gefächert ist und jeder dieser Gruppen andere Vorraussetzungen und Erfahrungen mit sich bringen, misslingt es oft eine benutzerfreundliche Oberfläche für alle Besucher bereitzustellen. Eine Usability-Analyse und -Optimierung ist hier notwendig, jedoch meist nicht möglich, da das Budget fehlt und die Möglichkeiten begrenzt sind. Das Ziel dieser Arbeit setzt genau an diesem Punkt an.\\
\\
Es soll ein Tool geschaffen werden, welches die eine Usability Evaluierung für beliebige Webseiten möglich macht. Dabei liegt jedoch das Hauptaugenmerk auf Hochschulseiten. Der Name der Applikation heißt \textit{thEvaluator}\footnote{Projektwebsite ist zu finden auf \url{http://qcentral.org})}. Im Gegensatz zu den Tools, die sich aktuell auf dem Markt etablieren konnten, wird thEvaluator einen aufgabenorientierten Ansatz bieten, bei dem jeder Testdurchgang genau beleuchtet werden kann. Testcases sind Grundlage der Usability Evaluierung. Sie enthalten Informationen über die Seite, die für den Test notwendig sind. Ebenfalls beinhaltet ein Testcase verschiedene  Tasks, die den User während des Aufenthalts leiten und genau Verhaltensmuster normaler Besucher abbilden sollen. Die Applikation selbst wird aus drei Komponenten bestehen. Grundlage zur Verwaltung und Persistierung der Daten bietet eine API, die sowohl REST-Services als auch Socket Streams annehmen wird. Als nächstes bietet eine Webapplikation die Möglichkeit Testcases anzulegen und zu evaluieren. Die wichtigste Komponente stellt die Browser Extension dar. Diese wird zur Durchführung der Tests zwingend benötigt und muss vorher in den Browser eingebunden werden. Aufgrund der begrenzten Bearbeitungszeit wird in der ersten Phase vorerst nur die Extension für den Chrome Browser entwickelt. Ebenfalls wird die Einführung in den Chrome Web Store nicht vorgesehen. Da sich die Laufzeitumgebung der Applikation im Browser befindet, fällt das Betriebssystem als Abhängigkeit rauß. Es wird lediglich eine Mindestversion für Chrome vorgeschrieben, die alle nötigen Webtechnologien, die für die Datenerfassung nötig sind, unterstützt.\\
\\
Als Abschlussziel des Projektes sind zwei Tests mit dem entwickelten Tool geplant. Der erste Test wird die Usability der Applikation selber testen. Dabei sollen Personen aus verschiedenen Zielgruppen eigene Testcases anlegen und sie selber durchführen. Die daraus gezogenen Resultate und Feedbacks werden genutzt, versteckte Fehler in der Usability und Funktionalität zu finden und das Tool zu verbessern. Im zweiten Test wird die Usability der Website der Beuth Hochschule\footnote{\url{http://www.beuth-hochschule.de/}} getestet. Ausgegangen wird von der Position eines Abiturienten, der Interesse an einem Studiengang der Fachhochschule hat. Da die Internetseite kaum interaktive Elemente bietet und mehr eine Informationsseite ist, liegt der Fokus der Aufgaben, die während des Tests gestellt werden, darauf, wichtige Information über das Studium zu finden und die dafür vorgesehenen Seiten zu öffnen. Der Test sollte Aufschluss über mögliche Usability-Schwächen geben und aufzeigen, dass wichtige Informationen schwer zu finden sind. Unter den Testpersonen befinden sich neben Studenten der Beuth Hochschule, die mit dem Aufbau der Seite vertraut sind, auch Studenten anderer Hochschulen, sowie Personen, die überhaupt nicht studiert haben. Ein Vergleich dieser Personengruppen bietet interessante Aufschlüsse und kann, in Form von Verbesserungsvorschlägen, der erste Ansatz zur Verbesserung der Usability der Seite sein.