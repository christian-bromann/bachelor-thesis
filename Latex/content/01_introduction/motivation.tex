%
% Motivation und Ausgangssituation
% Abschlussarbeit (Bachelor)
%
% Thema: Erstellung einer Browser Extension zur Usability Evaluierung von beliebigen Web-Applikationen über Heatmaps.
% Betreuer 1: Prof. Dr. Targo Pavlista
% Betreuer 2: Siamak Haschemi
%
% @author Christian Bromann <contact@christian-bromann.com>
%

\section{Motivation und Ausgangssituation}

Ein Werkzeug ist und bleibt unbenutzbar, solang dessen Bedienung nicht einleuchtend und verständlich ist. Daher ist auch der Erfolg einer Software immer von dessen Benutzbarkeit abhängig, ungeachtet dessen wie viel Zeit und Geld ins Design oder der Funktionalität geflossen sind. Das Internet entwickelt sich seit Jahren rasant weiter. Webseiten werden zu immer komplexeren Gebilden aus Bildern, Dokumenten und Funktionalität. Die Interaktivität und Bedienung dieser gleicht daher sehr der gewöhnlichen Programm-Software. Somit stehen Entwickler von Internetseiten ebenfalls vor der Herausforderung, diese für jeden Besucher gleich gut bedienbar zu gestalten. Die Tatsache wie schnell und bequem der Besucher einer Internetseite an eine gewünschte Information gelangt, entscheidet später darüber, ob er mit seinem Besuch auf der Seite zufrieden war oder nicht. Trifft dies zu, so kommt er gerne wieder, anderseits wird sich nach einer ähnlichen Seite umschauen.\\
\\
Der Einfluss von Usability auf Webseiten lässt sich hervorragend am Beispiel eines Webshops erklären, da im eCommerce Bereich der Erfolg am stärksten von der Bedienbarkeit des Online-Stores abhängt. Dieser muss den Kunden überzeugen können Online einen Einkauf zu tätigen, da nur bei einer erfolgreich abgeschlossen Transaktion der Betreiber der Seite Profit erzielen kann, auf das er angewiesen ist. Stimmt die Usability nicht, kommt es in den seltensten Fällen zu einer Kaufabwicklung. Das Ziel sollte es jedoch sein, ein angenehmes Gefühl bei dem Endkunden zu entwickeln, sodass er gewollt ist, den Shop wieder zu besuchen, um weitere Einkäufe zu tätigen. Die Qualität der Seite beginnt mit der detaillierten Darstellung von Produkten und endet mit einem übersichtlichen und transparenten Checkout. Alles muss genau aufeinander abgestimmt sein.\\
\\
\label{usability-auf-uniseiten}
Übersichtlichkeit ist eines von vielen Faktoren, die die Usability eines Produktes ausmachen. Ein weiteres Beispiel, wo dies eine bedeutende Rolle spielt, sind Universitäts-Webseiten, wie die der Beuth Hochschule. Sie beinhalten meist eine riesige Fülle von Informationen und sprechen einen großen Kreis von Menschen an. Diese haben jeweils sehr spezifische und konkrete Anforderungen an der Seite. Von Abiturienten, die sich für ein Studium bewerben wollen, bis hin zu Studenten, die sich über Lehrveranstaltungen und Sprechzeiten von Professoren informieren. Jede Information muss einfach und direkt erreichbar sein und das nicht nur für Desktop- sondern auch für Mobile-Auflösungen. Bei dieser breite an Besuchertypen eine sehr schwierige Aufgabe. Usability spielt deshalb hier eine enorm wichtige Rolle. Gerade für Neustudenten, die noch keine Erfahrungen im Unialltag haben und in eine komplett neue Welt eintauchen sind hilfreiche Informationen wichtig, um sich schnell zurechtzufinden. Leider gibt es heutzutage immer noch starken Bedarf an Optimierung, wenn es um die Informationsaufbereitung von Hochschulseiten geht. Studenten nutzen zwar verstärkt Online-Services von Hochschulen, finden sich jedoch meist nicht zurecht und haben ein unwohles Gefühl beim Nutzen dieses Angebotes. Oft schlägt sich das in langen Warteschlangen an der Studienberatung nieder und kann bis hin zu einer schlechten Unterrichtsvorbereitung führen. Da eine gezielte und professionelle Usability Evaluierung in Laboren viel zu teuer ist für das Budget einer Hochschule, werden diese Missstände selten behoben und Studenten und Professoren lernen damit zu leben. Diese Tatsache trug viel zu der Motivation dieser Arbeit bei.\\
\\
Dank Pionieren wie Jakob Nielsen oder Kasper Hornbæk, die bereits eine breite Liste von Usability Faktoren und Methoden erarbeitet und etabliert haben, muss dabei nicht von vorn begonnen werden. Schon durch einfache Testverfahren lassen sich Probleme erkennen, Verbesserungen evaluieren und die Benutzerfreundlichkeit steigern. Diese lassen sich schon mit wenig Aufwand mithilfe von Online Tools realisieren. Viele dieser Tools decken jedoch nur sehr spezielle Testfälle ab und erzeugen meist nur oberflächliche Resultate. Selten ist es möglich, das Verhalten eines einzelnen Benutzers zu analysieren, um seine Probleme zu erkennen und daraus die richtigen Schlüsse zu ziehen. Zudem sind die meisten Tools immer mit einer kostenpflichtigen Registrierung verbunden. Es gibt bisher auch kaum die Möglichkeit ein gewissen Nutzverhalten zu simulieren. Oft betrachten Usability Programme den gesamten User-Spektrum und bieten wenige Filtermechnismen an. Dies macht eine Analyse von gezielten Benutzergruppen oder speziellen Bereiche einer Anwendung unmöglich.\\
\\
Ein weiterer Punkt, der dieses Thema so spannend macht, ist das Erlernen von Usability-Aspekten als Webentwickler. In dieser Position ist es wichtig früh zu erkennen, wenn Module schlecht benutzbar sind und sie nicht ins Konzept der Seite passen. In diesem Fall ist die Kommunikation mit dem Projektmanager und die Reevaluierung des Konzepts erste Priorität, bevor eine Weiterentwicklung stattfindet. Dies kann am Ende viel Geld sparen und unnötige Korrekturschleifen mit dem Kunden vermeiden. Zudem führt die Entwicklung mit Usability-Aspekten im Hinterkopf von vornherein schon zu besseren Endresultaten. Hier ist es sicherlich nicht möglich alle Aspekte, die Benutzerfreundlichkeit ausmachen, abzudecken. Dennoch steigert es die Qualität der entwickelten Webanwendung automatisch.