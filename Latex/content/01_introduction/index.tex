%
% Einleitung
% Abschlussarbeit (Bachelor)
%
% Thema: Erstellung einer Browser Extension zur Usability Evaluierung von beliebigen Web-Applikationen über Heatmaps.
% Betreuer 1: Prof. Dr. Targo Pavlista
% Betreuer 2: Siamak Haschemi
%
% @author Christian Bromann <contact@christian-bromann.com>
%

% glossary entries 
\newglossaryentry{On-Demand}{name=On-Demand, description={(dts. auf Anforderung) Medium ist über ein Online Portal verfügbar über ein Stream oder via Download}}

\chapter{Einleitung}

Das World Wide Web ist eines der umfassendsten und wichtigsten Informationsquellen der Menschheit geworden. Es beeinflusst unser Denken und Handeln im Alltag sowie die Art miteinander zu kommunizieren. Laut dem aktuellsten Gutachten\cite{websurvey} von Netcraft\footnote{Netcraft ist ein Internetdienstleister, der neben Internet-Secruity auch Datenforschung und -analyse betreibt}, bei dem die Erreichbarkeit und Aktivität von Server getestet wurde, gibt es mittlerweile 629,939,191 erreichbare Webseiten im Internet, davon sind ca. ein drittel aktiv. Jedes Jahr kommen erneut mehrere Millionen Seiten dazu.\\
\\
Der Anreiz \glqq Online zu gehen\grqq{} ist dabei so vielfältig wie die Auswahl der Seiten. Die meisten Menschen nutzen das Internet zur Kommunikation, sei es das Chatten mit den besten Freund oder Freundin oder das Verbreiten einer 140 Zeichen langen Kurznachricht über Twitter. Neben den Social-Networks gibt es mittlerweile diverse andere Möglichkeiten, um mit Menschen aus der ganzen Welt in Kontakt zu treten oder sich als Person der Welt zu präsentieren. Weiterhin dient das Web mehr und mehr als Unterhaltungsmedium. Filme, Serien oder Musik sind \Gls{On-Demand} über verschiedene Plattformen erreichbar. Dies hilft und gleichzeitig schadet der Branche sehr, da das Austauschen von Daten immer leichter und ungefährlicher wird.\\
\\
Gehen wir in der Zeitleiste des Internets ein paar Jahrzehnte zurück und schauen auf die Ursprünge, wird klar, dass der eigentliche Nutzen dieser Technologie ein ganz anderer ist. Im Jahre 1969 entwickelte das US-Verteidigungsministeriums das ARPANET\footnote{Advanced Research Projects Agency Network}, welches erst für militärische Zwecke gedacht war und später der Vernetzung von Universitäten und Forschungseinrichtungen diente. Dieser primäre Zweck des Informationsaustausches führte zu der größten Veränderungen des Informationswesens seit der Erfindung des Buchdruckes durch Johannes Gutenberg. Während vor einem Jahrhundert für eine bestimmte Information ganze Bücher gelesen werden mussten, ist diese heutzutage nur noch wenige \glqq Klicks\grqq{} entfernt. Teilweise ist die Suche schon gar nicht mehr erforderlich, da wir im Alltag ständig und ungefiltert mit Informationen versorgt werden. Dies führt dazu, dass der Mensch sich zu viele Ereignisse und Nachrichten merken muss und ständig damit beschäftigt ist die Wichtigkeit der Information zu evaluieren. Oft geraten dabei wichtige Daten in Vergessenheit, da das Gehirn mit der Verarbeitung der Reize überfordert ist.\\
\\
Neben dem eigentlichen Informationsgehalt von Daten spielen Wichtigkeit und Aktualität eine wichtige Rolle, wenn es darum geht zu entscheiden, die Nachricht im Gehirn zu speichern oder nicht. Sie spiegeln den qualitativen Maßstab einer Nachricht wieder und helfen dem Menschen dabei diese richtig einzuordnen. Das Internet hat sich dazu entwickelt für den Nutzer immer die neuesten und relevantesten Daten für uns bereitzustellen. Dieses Qualitätsstreben macht nicht nur den Erfolg einer Suchmaschiene aus, sondern auch die einer einfachen Website.

%
% Motivation und Ausgangssituation
% Abschlussarbeit (Bachelor)
%
% Thema: Erstellung einer Browser Extension zur Usability Evaluierung von beliebigen Web-Applikationen über Heatmaps.
% Betreuer 1: Prof. Dr. Targo Pavlista
% Betreuer 2: Siamak Haschemi
%
% @author Christian Bromann <contact@christian-bromann.com>
%

\section{Motivation und Ausgangssituation}

Ein Werkzeug ist und bleibt unbenutzbar, solange dessen Bedienung nicht einleuchtend und verständlich ist. Daher ist auch der Erfolg einer Software immer von dessen Benutzbarkeit abhängig, ungeachtet dessen, wie viel Zeit und Geld ins Design oder der Funktionalität geflossen sind. Das Internet entwickelt sich seit Jahren rasant weiter. Webseiten werden zu immer komplexeren Gebilden aus Bildern, Dokumenten und Funktionalität. Die Interaktivität und Bedienung dieser gleicht daher sehr der gewöhnlichen Programm-Software. Somit stehen Entwickler von Internetseiten ebenfalls vor der Herausforderung, diese für jeden Besucher gleich gut bedienbar zu gestalten. Die Tatsache, wie schnell und bequem der Besucher einer Internetseite an eine gewünschte Information gelangt, entscheidet später darüber, ob er mit seinem Besuch auf der Seite zufrieden war oder nicht. Trifft dies zu, so kommt er gerne wieder, andererseits wird er sich nach einer ähnlichen Seite umschauen.\\
\\
Der Einfluss von Usability auf Webseiten lässt sich hervorragend am Beispiel eines Webshops erklären, da im eCommerce Bereich der Erfolg am stärksten von der Bedienbarkeit des Online-Stores abhängt. Dieser muss den Kunden überzeugen können Online einen Einkauf zu tätigen, da nur bei einer erfolgreich abgeschlossen Transaktion der Betreiber der Seite Profit erzielen kann, auf das er angewiesen ist. Stimmt die Usability nicht, kommt es in den seltensten Fällen zu einer Kaufabwicklung. Das Ziel sollte es jedoch sein, ein angenehmes Gefühl bei dem Endkunden zu entwickeln, sodass er den Shop gerne wieder besucht, um weitere Einkäufe zu tätigen. Die Qualität der Seite beginnt mit der detaillierten Darstellung von Produkten und endet mit einem übersichtlichen und transparenten Checkout. Alles muss genau aufeinander abgestimmt sein.\\
\\
\label{usability-auf-uniseiten}
Übersichtlichkeit ist einer von vielen Faktoren, die die Usability eines Produktes ausmachen. Ein weiteres Beispiel, wo dies eine bedeutende Rolle spielt, sind Universitäts-Webseiten, wie die der Beuth Hochschule. Sie beinhalten meist eine riesige Fülle von Informationen und sprechen einen großen Kreis von Menschen an. Diese haben jeweils sehr spezifische und konkrete Anforderungen an die Seite. Von Abiturienten, die sich für ein Studium bewerben wollen, bis hin zu Studenten, die sich über Lehrveranstaltungen und Sprechzeiten von Professoren informieren, greifen täglich viele verschiedene Benutzergruppen auf die Informationen der Seite zu. Diese müssen dabei einfach und direkt erreichbar sein und das nicht nur für Desktop- sondern auch für Mobile-Auflösungen. Eine ergonomische Benutzerführung ist daher hier sehr bedeutsam. Gerade für Neustudenten, die noch keine Erfahrungen im Unialltag haben und in eine komplett neue Welt eintauchen, sind hilfreiche Informationen wichtig, um sich schnell zurechtzufinden. Leider gibt es heutzutage immer noch starken Bedarf an Optimierung, wenn es um die Informationsaufbereitung von Hochschulseiten geht. Studenten nutzen zwar verstärkt Online-Services von Hochschulen, finden sich jedoch meist nicht zurecht und haben ein unwohles Gefühl beim Nutzen der Angebote. Oft schlägt sich das in langen Warteschlangen an der Studienberatung nieder und kann bis hin zu einer schlechten Unterrichtsvorbereitung führen. Da eine gezielte und professionelle Usability Evaluierung in Laboren viel zu teuer für das Budget einer Hochschule ist, werden diese Missstände selten behoben und Studenten und Professoren lernen damit zu leben. Diese Tatsache trug viel zu der Motivation dieser Arbeit bei.\\
\\
Dank Pionieren wie Jakob Nielsen oder Kasper Hornbæk, die bereits eine breite Liste von Usability Faktoren und Methoden erarbeitet und etabliert haben, muss bei der Optimierung nicht von vorn begonnen werden. Schon durch einfache Testverfahren lassen sich Probleme erkennen, Verbesserungen evaluieren und die Benutzerfreundlichkeit steigern. Diese lassen sich schon mit wenig Aufwand mithilfe von Online Tools realisieren. Viele dieser Tools decken jedoch nur sehr spezielle Testfälle ab und erzeugen meist nur oberflächliche Resultate. Selten ist es möglich, das Verhalten eines einzelnen Benutzers zu analysieren, um seine Probleme zu erkennen und daraus die richtigen Schlüsse zu ziehen. Zudem sind die meisten Tools immer mit einer kostenpflichtigen Registrierung verbunden. Es gibt bisher auch kaum die Möglichkeit, ein gewisses Nutzverhalten zu simulieren. Oft betrachten Usability Programme den gesamten User-Spektrum und bieten wenige Filtermechnismen an. Dies macht eine Analyse von gezielten Benutzergruppen oder speziellen Bereichen einer Anwendung unmöglich.\\
\\
Ein weiterer Punkt, der dieses Thema so spannend macht, ist das Erlernen von Usability-Aspekten als Webentwickler. In dieser Position ist es wichtig, früh zu erkennen, wenn Module schlecht benutzbar sind und sie nicht ins Konzept der Seite passen. In diesem Fall ist die Kommunikation mit dem Projektmanager und die Reevaluierung des Konzepts erste Priorität, bevor eine Weiterentwicklung stattfindet. Dies kann am Ende viel Geld sparen und unnötige Korrekturschleifen mit dem Kunden vermeiden. Zudem führt die Entwicklung mit Usability-Aspekten im Hinterkopf von vornherein schon zu besseren Endresultaten. Hier ist es sicherlich nicht möglich alle Aspekte, die Benutzerfreundlichkeit ausmachen, abzudecken. Dennoch steigert es die Qualität der entwickelten Webanwendung automatisch.
%
% Projektziel
% Abschlussarbeit (Bachelor)
%
% Thema: Erstellung einer Browser Extension zur Usability Evaluierung von beliebigen Web-Applikationen über Heatmaps.
% Betreuer 1: Prof. Dr. Targo Pavlista
% Betreuer 2: Siamak Haschemi
%
% @author Christian Bromann <contact@christian-bromann.com>
%

\section{Projektziel}

Hochschulseiten haben es schwer, die Fülle an Informationen optimal auf einer Website zu platzieren oder zu verlinken. Da die Zielgruppe der Besucher von Abiturabsolventen bis hin zu Professoren und Alumni sehr breit gefächert ist und jeder dieser Gruppen andere Vorraussetzungen und Erfahrungen mit sich bringen, misslingt es oft eine benutzerfreundliche Oberfläche für alle Besucher bereitzustellen. Eine Usability-Analyse und -Optimierung ist hier notwendig, jedoch meist nicht möglich, da das Budget fehlt und die Möglichkeiten begrenzt sind. Das Ziel dieser Arbeit setzt genau an diesem Punkt an.\\
\\
Es soll ein Tool geschaffen werden, das die Usability-Evaluierung für beliebige Webseiten möglich macht. Dabei liegt jedoch das Hauptaugenmerk auf Hochschulseiten. Der Name der Applikation heißt \textit{thEvaluator}\footnote{Projektwebsite ist zu finden auf \url{http://qcentral.org}}. Im Gegensatz zu den Tools, die sich aktuell auf dem Markt etablieren konnten, wird \textit{thEvaluator} einen aufgabenorientierten Ansatz bieten, bei dem jeder Testdurchgang genau beleuchtet werden kann. Testcases sind Grundlage der Usability-Evaluierung. Sie enthalten Informationen über die Seite, die für den Test notwendig sind. Ebenfalls beinhaltet ein Testcase verschiedene  Tasks, die den User während des Aufenthalts leiten und die Verhaltensmuster normaler Besucher abbilden sollen. Die Applikation selbst wird aus drei Komponenten bestehen. Grundlage zur Verwaltung und Persistierung der Daten ist eine API, die sowohl REST-Services als auch Socket Streams unterstützen wird. Als Nächstes bietet eine Webapplikation die Möglichkeit, Testcases anzulegen und zu evaluieren. Die wichtigste Komponente stellt die Browser Extension dar. Diese wird zur Durchführung der Tests zwingend benötigt und muss vorher in den Browser eingebunden werden. Aufgrund der begrenzten Bearbeitungszeit wird in der ersten Phase die Extension vorerst nur für den Chrome Browser entwickelt. Ebenfalls wird die Einführung in den Chrome Web Store nicht vorgesehen. Da sich die Laufzeitumgebung der Applikation im Browser befindet, fällt das Betriebssystem als Abhängigkeit rauß. Es wird lediglich eine Mindestversion für Chrome vorgeschrieben, die alle nötigen Web-Technologien, die für die Datenerfassung nötig sind, unterstützt.\\
\\
Als Abschlussziel des Projektes sind zwei Tests mit dem entwickelten Tool geplant. Der erste Test wird die Usability der Applikation selber testen. Dabei sollen Personen aus verschiedenen Zielgruppen eigene Testcases anlegen und sie selber durchführen. Die daraus gezogenen Resultate und Feedbacks werden genutzt, um versteckte Fehler in der Usability und Funktionalität zu finden und das Tool zu verbessern. Im zweiten Test wird die Benutzerfreundlichkeit der Webseite der Beuth Hochschule\footnote{\url{http://www.beuth-hochschule.de/}} getestet. Ausgegangen wird von der Position eines Abiturienten, der Interesse an einem Studiengang der Fachhochschule hat. Da die Internetseite kaum interaktive Elemente bietet und mehr den Charakter einer Informationsseite besitzt, liegt der Fokus der Aufgaben, die während des Tests gestellt werden, darauf, wichtige Information über das Studium zu finden und die dafür vorgesehenen Seiten zu öffnen. Der Test soll Aufschluss über mögliche Usability-Schwächen geben und aufzeigen, dass wichtige Informationen schwer zu finden sind. Unter den Testpersonen befinden sich neben Studenten der Beuth Hochschule, die mit dem Aufbau der Seite vertraut sind, auch Studenten anderer Hochschulen, sowie Personen, die überhaupt nicht studiert haben. Ein Vergleich dieser Personengruppen bietet interessante Aufschlüsse und kann, in Form von Verbesserungsvorschlägen, der erste Ansatz zur Steigerung der Usability der Seite sein.
%
% Aufbau der Arbeit
% Abschlussarbeit (Bachelor)
%
% Thema: Erstellung einer Browser Extension zur Usability Evaluierung von beliebigen Web-Applikationen über Heatmaps.
% Betreuer 1: Prof. Dr. Targo Pavlista
% Betreuer 2: Siamak Haschemi
%
% @author Christian Bromann <contact@christian-bromann.com>
%

\section{Aufbau der Arbeit}

Die Klärung des Begriffes \textit{Usability} und \textit{Benutzerfreundlichkeit} ist Mittelpunkt des ersten Teil der Arbeit. Der Focus der Erläuterungen konzentriert sich dabei auf das Web. Da das Thema Usability ein weites Spektrum umfasst und es für jeden Teilbereich der Informatik, neben den allgemeinen Regeln, spezielle Anforderungen stellt, ist es wichtig hier zu differenzieren. Auf Internetseiten bestimmen andere Einflüsse die Benutzbarkeit des Produktes als bei Softwareprogrammen oder Hardwaregeräten. Bevor dann auf die Applikation eingegangen wird, analysiert die Arbeit vorher schon vorhandene und etablierte Produkte auf dem Markt, die sich ebenfalls mit Usability-Evaluierungen von Webseiten beschäftigen.\\
\\
Das nächste Kapitel beschäftigt sich voll und ganz mit der Hauptanwendung \textit{thEvaluator}. Zum Anfang gibt es einen genauen Überblick über dessen Funktionen und Einsatzmöglichkeiten. Eine Beschreibung der Zielbestimmungen informiert zudem über wichtige Entscheidungen über die Konzeption der Anwendung und gibt einen Einblick darüber, warum bestimmte Technologien eingesetzt werden. Im weiteren Verlauf wird dann jede Komponente genau erläutert. Von dessen Aufbau und Architektur bis hin zur Funktionsweise bekommt der Leser einen genauen Einblick in die entwickelte Software. Verschiedene technische Details werden erläutert sowie Probleme im Entwicklungsprozess und dessen Lösungen beschrieben. Da der Technologie-Stack auf den neuesten Web-Frameworks und APIs basiert, gibt es ebenfalls einen genauen Einblick in die moderne Frontend-Entwicklung von heute. Am Ende des Kapitels werden abschließend Möglichkeiten zur Erweiterungen der Software aufgezeigt.\\
\\
Der letzte Teil der Arbeit enthält die Auswertung über die Usability-Tests, die mit der Anwendung durchgeführt werden. Dabei wird das Tool an sich sowie die Webseite der Beuth Hochschule einem Test unterzogen. Teilnehmer sind neben Studenten der Fachhochschule auch Personen aus komplett anderen Fachrichtungen. Nach der Evaluierung der Ergebnisse werden Schwachstellen aufgezeigt und Verbesserungsmöglichkeiten erörtert.