%
% Aufbau der Arbeit
% Abschlussarbeit (Bachelor)
%
% Thema: Erstellung einer Browser Extension zur Usability Evaluierung von beliebigen Web-Applikationen über Heatmaps.
% Betreuer 1: Prof. Dr. Targo Pavlista
% Betreuer 2: Siamak Haschemi
%
% @author Christian Bromann <contact@christian-bromann.com>
%

\section{Aufbau der Arbeit}

Die Klärung des Begriffes \textit{Usability} und \textit{Benutzerfreundlichkeit} ist Mittelpunkt des ersten Teil der Arbeit. Der Focus der Erläuterungen konzentriert sich dabei auf das Web. Da das Thema ein weites Spektrum erfasst und es für jeden Teilbereich der Informatik, neben den allgemeinen Regeln, spezielle Anforderungen stellt, ist es wichtig hier zu differenzieren. Auf Internetseiten bestimmen andere Einflüsse die Benutzbarkeit des Produktes als bei Softwareprogrammen oder Hardwaregeräten. Bevor es dann zur Applikation selber kommt, analysiert die Arbeit vorher schon vorhandene und etablierte Produkte auf dem Markt, die sich ebenfalls mit Usability Evaluierungen von Webseiten beschäftigen.\\
\\
Das nächste Kapitel beschäftigt sich voll und ganz mit der Hauptanwendung \textit{thEvaluator}. Zum Anfang gibt es einen genauen Überblick über dessen Funktionen und Einsatzmöglichkeiten. Eine Beschreibung der Zielbestimmungen informiert zudem über wichtige Entscheidungen der Nutzungsart und gibt einen Einblick darüber warum am Ende die Software so funktioniert und bedient werden muss und nicht anders. Im weiteren Verlauf wird dann jede Komponente genau erläutert. Von dessen Aufbau und Architektur bis hin zur Funktionsweise bekommt der Leser einen genauen Einblick in die entwickelte Software. Gründe für den Einsatz verschiedener Technologien werden erläutert sowie Probleme im Entwicklungsprozess und dessen Lösungen beschrieben. Da der Technologie-Stack auf den neuesten Frameworks und APIs basiert, gibt es ebenfalls einen genauen Einblick in die moderne Frontend-Entwicklung von heute. Am Ende des Kapitels werden abschließend Möglichkeiten zur Erweiterungen der Software aufgezeigt.\\
\\
Der letzte Teil der Arbeit enthält die Auswertung über die Usability Tests, die mit der Anwendung durchgeführt werden. Dabei wird das Tool an sich einem Test unterzogen, sowie die Webseite der Beuth Hochschule. Teilnehmer sind neben Studenten der Fachhochschule auch Personen aus komplett anderen Fachrichtungen. Nach Evaluierung der Ergebnisse werden Schwachstellen aufgezeigt und Verbesserungsmöglichkeiten erörtert.