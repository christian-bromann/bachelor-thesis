%
% Marktanalyse
% Abschlussarbeit (Bachelor)
%
% Thema: Erstellung einer Browser Extension zur Usability Evaluierung von beliebigen Web-Applikationen über Heatmaps.
% Betreuer 1: Prof. Dr. Targo Pavlista
% Betreuer 2: Siamak Haschemi
%
% @author Christian Bromann <contact@christian-bromann.com>
%

\section{Marktanalyse}

Die Durchführung von Usability Tests geschieht vornehmlich in Laboren und ist meist mit hohen Kosten verbunden. Eine Online-Agentur oder ein Startup kann diese Kosten nicht aufbringen. Diese Tests werden oft deshalb gar nicht durchgeführt und bleiben lediglich für profitbringende Webseiten mit Werbung eine Option. Die einzige Möglichkeit, die Usability der eigenen Seite halbwegs genau zu testen, ist der Verzicht auf Labortests und die Nutzung von Onlinetools. Diese gibt es mittlerweile in verschiedensten Varianten. Nur wenige von ihnen sind kostenlos. In diesem Kapitel werden drei Tools von unterschiedlicher Funktionsbreite vorgestellt und verglichen.

\subsection*{CrazyEgg\footnote{\url{http://www.crazyegg.com/}}}

Dieses sehr populäre Tool zeichnet alle Klicks der User einer Website auf und ist in der Lage, die Daten in den verschiedenster Formen informativ zu visualisieren. Neben den gewöhnlichen Attention- und Clickmaps, bietet das Feature \textit{Confetti} die Möglichkeit, hinter jedem Klick die Herkunft des Besuchers festzustellen. Das Tool ist zudem in der Lage, jegliche Elemente auf der Seite, wie z. B. Links, Bilder oder Überschriften, zu erkennen und die Rate der Klicks dieser anzuzeigen. Eingebunden wird es durch das Einfügen eines JavaScript Codeschnipsels auf der eigenen Seite. Die Auswertung erfolgt auf der Toolseite und ist als Excel Datei exportierbar.

\subsubsection*{Vorteile}
Es bietet trotz des günstigen Preises eine Reihe von gebräuchlichen Visualisierungen, die dabei helfen können, die Usability der Seite zu steigern.

\subsubsection*{Nachteile}
Das Tool analysiert lediglich das Klick- und Scrollverhalten der User für eine individuelle Seite. Daraus lassen sich keine Schlüsse auf den Gesamtkontext des Besuches ziehen. Es ist dadurch schwer herauszufinden, warum der User geklickt hat.

\subsubsection*{Preis}
Angeboten werden vier verschieden Pläne von 9\$ bis hin zu 99\$.

\subsection*{Loop$^{11}$ \footnote{http://www.loop11.com/}}

Loop$^{11}$ ist ein taskbbasiertes Usability-Testing-Tool. Die Tests werden von ausgewählten Probanden durchgeführt und müssen dabei nicht durch den Seitenbetreiber moderiert werden. Dieser muss lediglich einen Testcase mit verschiedenen Aufgaben erzeugen und ihn an ausgewählte User schicken. Die Probanden werden dann mit einer Aufgabe konfrontiert, entscheiden aber selbstständig, wann sie diese erfüllt haben oder nicht. In der Auswertung ist anschließend die Erfolgsrate und die dafür benötigte Zeit einsehbar. Zudem ist es möglich, Besucherpfade zu analysieren. Das Tool benötigt dabei keine Software. Die getestete Seite wird in einem eigenen iFrame eingebunden und analysiert. Jede frei zugängliche Seite des Internets ist somit testbar.

\subsubsection*{Vorteile}
Es muss kein Code implementiert werden, wodurch es möglich ist, jede beliebige Seite des Internets zu testen.

\subsubsection*{Nachteile}
Die Probanden können selbst entscheiden, wann die Aufgabe gelöst ist. Die Entscheidung des Users kann im Nachhinein nicht erfragt oder ausgewertet werden.

\subsubsection*{Preis}
Für die Nutzung des Tools werden Lizenzen für 158\$ bis hin zu 825\$ pro Monat verkauft.

\subsection*{Mouse Eye Tracking\texttrademark \footnote{http://met.picnet.com.au/}}

Das mit zwei \textit{iAwards} ausgezeichnete Tool \textit{Mouse Eye Tracking\texttrademark} bietet neben Click- und Heatmaps viele nützliche Visualisierungen vom Verhalten der User an. Es ist sogar möglich, die Mausposition eines einzigen Besuchers über den kompletten Zeitraum des Aufenthaltes auf der Seite zu verfolgen. Ebenfalls werden Besucherpfade gespeichert und analysiert. Eingebunden wird es standardgemäß mit dem Einfügen von JavaScript Code im HTML.

\subsubsection*{Vorteile}
Die Analyse des Nutzerverhaltens ist sehr granular. Für jeden einzelnen Benutzer ist die Mausbewegung verfolgbar.

\subsubsection*{Nachteile}
Die Aufzeichnung der Daten erfolgt ohne jegliche Bedingungen, die der Benutzer, in Form von z.B. Tasks, erfüllen muss. Seine Intention und Verhaltensweisen sind dadurch schwer einschätzbar.

\subsubsection*{Preis}
Nach einer 7-tägigen Probephase kann der Kunde ein Paket für 10\$ oder 20\$ pro Monat wählen.















