%
% Kapitel: Usability
% Abschlussarbeit (Bachelor)
%
% Thema: Erstellung einer Browser Extension zur Usability Evaluierung von beliebigen Web-Applikationen über Heatmaps.
% Betreuer 1: Prof. Dr. Targo Pavlista
% Betreuer 2: Siamak Haschemi
%
% @author Christian Bromann <contact@christian-bromann.com>
%

\chapter{Usability}

Mit dem Substantiv \textit{Gebrauchstauglichkeit} übersetzt die ISO Norm 9241 den Begriff Usability und bezeichnet damit den Prozess der \glqq Benutzer-orientierte Gestaltung interaktiver Systeme\grqq{} \cite{iso9241}. Oft gleichgesetzt mit dem Fachgebiet der Human-Computer-Interaktion, beschreibt es das Bestreben dem Benutzer einer Software zu helfen, ein gewisses Ziel möglichst schnell und effizient zu erreichen. Es ist damit der wichtigste Erfolgsfaktor der Systemqualität. Die Bedienung des Computers ist dabei immer mit einer Aufgabe verbunden, die der Nutzer mithilfe einer Software versucht zu lösen. Dieses Prinzip lässt sich vereinfacht in ein Mensch-Maschinen-System zerlegen. Es basiert auf vier prinzipiellen Komponenten, die möglichst ergonomisch miteinander agieren sollen.

\vspace{0.6cm}
\includegraphics[scale=0.5]{./images/human-machine-system}
\begin{figure}[htb]
   \centering
   \caption{Mensch-Maschinen-System \textbf{Quelle:} \cite{UsabilityKompakt}}
    \label{mms}
\end{figure}

Die Eigenschaften und Funktionen einer Software bestimmen dabei nicht ausschließlich dessen Usability. So spielt das Einsatzumfeld ebenfalls eine wichtige Rolle. Beispielhaft ausgedrückt kann ein Flaschenöffner noch so handlich sein, wenn er dafür genutzt wird, Nägel in die Wand zu schlagen. Daher liegt eine benutzerfreundliche und ergonomische Software erst dann vor, wenn \glqq sie von den vorhergesehen Benutzern einfach erlernt und effizient verwendet werden [kann] und diese damit ihre beabsichtigen Ziele und Aufgaben zufriendenstellend [ausführt]\grqq{} \cite{UsabilityKompakt}.\\
\\
Um dies zu erreichen, sollte \textit{Usability Engineering} ein wichtiger Prozess in der Entwicklung einer Software sein. Angefangen von der Definition messbarer Usability Attribute über die Festlegung von numerischen Werten dieser bis hin zum Testen der Software ist dies ein sich immer wiederkehrendes Verfahren, bei dem sich die Methodik flexibel gestaltet. Die Attributen sollten möglichst einfache messbare Eigenschaften der Software darstellen und nach jedem Test mit den festgelegten Werten verglichen werden. Sind die Werte nicht erreicht worden, folgt eine Analyse der aufgetretenen Probleme und die anschließende Verbesserung der Software. Es handelt sich um einen iterativen Prozess, der nie wirklich einen Abschluss findet. Verbesserungen in der Usability sind immer möglich. Meist erreicht man jedoch einen zufriedenstellenden Grad der Benutzerfreundlichkeit oder hat keine Zeit oder Budget mehr für weitere Tests.\\
\\
\textit{Usability Testing} bezeichnet den gesamten Ablauf, bei dem überprüft wird, ob festgelegte Usability Ziele erreicht wurden oder nicht. Die Tests unterteilen sich in verschiedene Methoden, die unterschiedliche Daten für die heuristische Auswertung sammeln. Diese sollten in einer Vorbereitungsphase genau bestimmt werden und gewissen Kriterien unterliegen, damit sie ein genaues Abbild des Nutzerverhaltens widerspiegeln. Die Durchführung muss dabei nicht zwingend in einem Labor erfolgen. Oft verfälschen sogar Tests unter Laborbedingungen das Ergebnis, da der Benutzer der Software nicht in einem gewohntem Umfeld agiert und dadurch zu einem anderen Verhalten neigt. Eine festgeschrieben Mindestanzahl von Personen, die an dem Test teilnehmen, gibt es nicht. Nach Jakob Nielsen reichen schon 5 Probanden aus \cite{anzahlTestpersonen}, um einen großen Anteil der Probleme herauszufinden. Eine anschließende Analyse der Ergebnisse durch verschiedene Metriken deckt Schwachstellen in der Benutzung auf und kann mögliche Verbesserungen der Software herleiten.\\
\\
Im Laufe der letzten Jahrzehnte wurde für die Softwareentwicklung verschiedene Vorgehensmodelle entwickelt um eine koordinierte Vorgehensweise für die Erstellung von Software zu gewährleisten. Modelle, wie zbs. das Wasserfallmodell oder das V-Modell, werden in der Realität jedoch kaum wirklich eingehalten. Es enstehen immer wieder Mischformen, bei denen Elemente der benutzerorientierten Entwicklung integriert werden. Dieser, als sogenannte \textit{Usability Engineering Lifecycle} bezeichnete, Prozess bringt nach der Analyse- und Entwicklungsphase noch eine eine Nutzungsphase mit sich, da davon ausgegangen wird, dass \glqq trotz sorgfältiger Analyse und Entwicklung einige Nutzungsprobleme [sich] erst im Echtbetrieb herausstellen\grqq{} \cite{lifecycle}.

%
% Usability I'm Web
% Abschlussarbeit (Bachelor)
%
% Thema: Erstellung einer Browser Extension zur Usability Evaluierung von beliebigen Web-Applikationen über Heatmaps.
% Betreuer 1: Prof. Dr. Targo Pavlista
% Betreuer 2: Siamak Haschemi
%
% @author Christian Bromann <contact@christian-bromann.com>
%

\section{Usability im Web}
Das Thema Usability beschäftigt viele Bereiche der IT-Branche, doch nirgends erfährt sie so viel Aufmerksamkeit wie im Web. Hier ist eine hohe Benutzerfreundlichkeit essenziell, da der Besucher sofort die Webseite verlässt, wenn ihn etwas stört. Nirgendwo anders bestimmt das Wohlgefühl des Nutzers über die Dauer der Nutzung eines Produktes. Es ist der wichtigste Erfolgsfaktor eine Seite. Nicht nur deshalb beschäftigen sich so viele Menschen mit dem Thema Usability im Web. Es gibt mittlerweile dutzende Richtlinien und Standards, die verschiedene Regeln für Design, Struktur und Inhalt vorgeben. Einer der größten sind die \glqq Research-Based Web  Design \& Usability Guidelines\grqq{}, veröffentlicht vom U.S. Department of Health and Human Services, die ISO Norm 9241 sowie die Studie der JISC (JointInformation Systems Committee for higher education) über \glqq Guidelines For Academic Web Sites\grqq{} \cite{isoStandards}. Jedes Jahr tauschen sich Experten auf vielen Konferenzen über das Themas Usability aus und erhalten Preise von etlichen Usability Award Stiftungen. Dies zeigt wie groß der Focus auf diese Thematik liegt und unterstreicht ihre Wichtigkeit.\\
\\
Nach Jakob Nielsen bezieht sich Usability auf Eigenschaften wie Erlernbarkeit, Effizienz, Einprägsamkeit, Fehlerbehandlung und Nutzerzufriedenheit. Projiziert man dies auf das Web, so müssen die Definition der Attribute etwas verfeinert werden, um sie auf Webapplikationen anwenden zu können. Der Grund des Besuches einer Webseite basiert entweder auf das Suchen von bestimmten Informationen oder die Nutzung von Services, die dort angeboten werden. Die Applikation muss dabei das Vorhaben des Besuchers unterstützten, indem sie durch klare Trennung von Inhalt und Navigation leicht zu erlernen ist, ihre Inhalte effizient und schnell zugänglich macht und nach langer Abwesenheit immer noch nutzbar ist. Zudem sollte sie bei einem Fehler dem User Hilfe anbieten und ihm eine Art Wohlgefühl bei der Nutzung bereiten \cite{UsabilityPriciples}.\\
\\
Aus diesen Richtlinien lassen sich viele einzelne Regeln erstellen, die bei der Entwicklung Beachtung finden sollten. Diese können in vier grobe Bereiche eingeteilt werden: Zugänglichkeit, Identität, Navigation und Inhalt. Jeder dieser Regeln sind nicht zwingend verpflichtend und treffen auch nicht immer für jede Seite zu.


\subsection{Zugänglichkeit}

Abgesehen vom eigentlichen Inhalt und Aussehen ist die Zugänglichkeit der erste Faktor, der die Usability einer Seite bestimmt. Sie ist Maßstab dafür, wie gut der Benutzer auf die Inhalte zugreifen kann. Dies beginnt mit der Ladezeit einer Seite. Die meisten Menschen surfen heutzutage mit einer schnellen DSL Verbindung und sehen sie Seite schon innerhalb von wenigen Sekunden. Braucht eine Seite zu lange, um die Inhalte anzuzeigen, so verliert der User schnell die Lust zu warten und besucht eine andere Seite. Da Webseiten immer medialer ausgestattet werden, muss der Browser immer mehr Daten laden. Es ist deshalb wichtig, dass nicht alles auf einmal, sondern datenlastige Elemente nachzuladen. So bekommt der Besucher schon nach kurzer Zeit etwas zu sein und muss nicht Ewigkeiten auf einen weißen Bildschirm starren. Eine weitere Möglichkeit, die große Datenflut in den Griff zu bekommen, ist, Bilder und eingebundene Scripte und Styles zu komprimieren, um sie als ein komplettes Paket an den Browser zu schicken.\\
\\
Sind die Inhalte einmal geladen, so müssen sie auch gut zu erkennen sein. Ein gesunder Kontrast zwischen Hintergrund und Text sollte daher immer vorliegen. Des Weiteren ist eine ausreichende Schriftgröße wichtig, um die Lesbarkeit des Textes zu gewährleisten. Diese hängt stark davon ab, auf welche Zielgruppe sich die Seite konzentriert. Um sicher zu gehen ist es von Vorteil dem Besucher eine Möglichkeit zu geben, die Schriftgröße manuell zu ändern.\\
\\
Ein weiteres Hindernis sind Zusatzkomponenten, die eine Seite benötigt, um ihren Inhalt darstellen darstellen zu können. Das bekannteste Beispiel hierfür ist Flash. Obwohl die Mehrheit der Besucher dieses Zusatzprogramm installiert haben, gibt es dennoch viele User, die dies nicht haben. Gerade Leute, die mit einem Smartphone ins Internet gehen, müssen auf diese Inhalte verzichten. Dabei gibt es oft keinen Grund externe Programme in den Browser einzubinden. Der heutige Standard rund um HTML, JavaScript und CSS bietet alle Möglichkeiten, eine gleichwertige Applikation ohne die Erweiterung durch Flash oder Silverlight zu erstellen.\\
\\
Das Thema Internetzugang über das Smartphone ist ebenfalls eine wichtige Thematik. Immer mehr Menschen surfen im Internet während sie unterwegs sind oder bequem über das Tablett auf dem Sofa. Heutzutage sind noch nicht alle Webseiten dafür optimiert. Die Darstellung auf den deutliche kleineren Bildschirmen ist meist nicht für die Nutzung dafür geeignet. Mit einem \textit{Responsive Design} kann eine Seite speziell für eine kleinere Auflösung ausgeliefert werden und bietet auch den mobilen Nutzer eine komfortable und benutzerfreundliche Nutzung.


\subsection{Identität}

Das Erste was dem User durch den Kopf schießt sobald er eine Webseite aufruft ist: \textit{\glqq Wer oder Was stellt sich hier vor?\grqq{}} oder \textit{\glqq Was wird hier geboten?\grqq{}}. Um relativ schnell das Vertrauen des Benutzers zu gewinnen, sollte das Corperate Design schnell erkennbar sein, bspw. durch ein prominente Position des Logos in der oberen Hälfte der Seite. Für eine schnelle Übersicht ist zudem eine kurze und klare Beschreibung über den Inhalt und den Service direkt beim Start von Vorteil.\\
\\
Eine Kontaktseite sollte ebenfalls immer direkt erreichbar sein. Gerade bei Seiten, die einen Service anbieten ist dies enorm wichtig. Kommt es zu Fragen, da zbs. irgendetwas nicht richtig funktioniert, suchen Besucher immer die Möglichkeit auf eine Kontaktaufnahme und sind enttäuscht, wenn sie keinen finden.


\subsection{Navigation}

Sobald der Besucher erst einmal weiß, von wem er die Seite besucht und was ihn hier erwartet, braucht er eine Navigation, die ihn an sein Ziel führt. Diese sollte konsistent und klar zu erkennen sein. Eine Hauptnavigation, die auf jeder Seite an der gleichen Position zu finden ist, führt den Besucher am einfachsten durch die Struktur der Seite.\\
\\
Ist diese Struktur jedoch sehr komplex, führt sie oft zu Orientierungsverlust und überfordert den Besucher. Dies ist oft bei akademischen Webseiten der Fall. Hier werden verschiedenste Inhalte für ganz unterschiedliche Benutzergruppen angeboten. Diese haben es schwer, die gewünschten Informationen zu finden, da der Content tief verschachtelt in diversen Oberkategorien steckt. Die Erstellung einer geeigneten Link Struktur ist deshalb eines der wichtigsten und gleichzeitig schwersten konzeptionellen Aufgaben. Die Analyse von Verhaltens- und Navigationsmustern und der anschließenden Verbesserung der Link Struktur einer Seite fällt unter den Begriff des \textit{Web-Usage-Mining}, welches ein Untersuchungsgegenstand des \textit{Web Minings} ist. \cite{webusagemining}.

\begin{equation}
    E = \sum_{i != j} \frac{C_{ij}}{[N * (N - 1)]}
    \label{connectivity}
\end{equation}

\subsection{Inhalt}









%
% Einflüsse
% Abschlussarbeit (Bachelor)
%
% Thema: Erstellung einer Browser Extension zur Usability Evaluierung von beliebigen Web-Applikationen über Heatmaps.
% Betreuer 1: Prof. Dr. Targo Pavlista
% Betreuer 2: Siamak Haschemi
%
% @author Christian Bromann <contact@christian-bromann.com>
%

\section{Einflüsse}

% Was kann die Benutzbarkeit beeinflussen
% Barrierefreiheit
% X-Browser Gleichheit
%
% Metriken
% Abschlussarbeit (Bachelor)
%
% Thema: Erstellung einer Browser Extension zur Usability Evaluierung von beliebigen Web-Applikationen über Heatmaps.
% Betreuer 1: Prof. Dr. Targo Pavlista
% Betreuer 2: Siamak Haschemi
%
% @author Christian Bromann <contact@christian-bromann.com>
%

\newglossaryentry{Fixationen}{name=Fixationen, description={Als Fixationen werden sogenannte \textit{Eye-Stops} (Augenanhaltepunkte) bezeichnet.}}

\section{Metriken und Analyse}
\label{metrics}

Usability ist keine Eigenschaft, die einfach abzulesen ist. Sie ist ebenfalls nicht exakt bestimmbar und lässt sich ebenfalls nicht in numerischen Werten ausdrücken. Um sie herauszufinden, gibt es zwei Möglichkeiten. Entweder man vertraut seinen eigenem Gefühl beim Benutzen der Software oder erstellt Tests und nutzt Metriken zur Auswertung der Ergebnisse. Die Intuition ist wichtig und kann im ersten Moment viel aussagen, bietet jedoch keine wirkliche Sicherheit, da sie lediglich die eigene Meinung repräsentiert. Da eine Software von verschiedenen Personen mit unterschiedliche Ansprüche benutzt wird, reicht eine wage Vermutung nicht aus.\\
\\
Eine Usability-Studie bringt ein allgemeineres Abbild der Benutzbarkeit. Die aus den Tests erzeugten Daten lassen sich durch verschiedenste Metriken analysieren und auswerten. Sie unterliegen dabei keinen Richtlinien und spezialisieren sich teilweise auf bestimmte Aspekte bei der Benutzung einer Software. Abhängig davon, welche Aspekte analysiert und welche Daten dabei aufgezeichnet werden, erfolgt die Erstellung von Metriken ganze individuell für jeden Test. Sie bilden das Maß der Benutzbarkeit und sollten objektiv bestimmbar und vergleichbar sein.\\
\\
Usability-Tests lassen sich besser vergleichen, wenn Probanden Aufgaben während der Nutzung erledigen müssen. Metriken können dazu die Kriterien bieten. So kann bspw. verglichen werden, wie viel User eine bestimmte Aufgabe erfolgreich gelöst haben, wie lange sie dafür brauchten, wie viele Fehlermeldungen gezeigt wurden oder wie Zufrieden der User mit der Nutzung war. Hier reichen laut Nielsen \cite{metrics} schon 5 Personen aus, um einen aussagekräftigen Vergleich zu bekommen. Durch eine Gewichtung der Metriken oder der einzelnen Resultate kann sich der Test auf bestimmte Faktoren der Usability konzentrieren. So ist die Performance in manchen Situationen wichtiger als das Aussehen einer Seite. Bei der Auswahl der Testpersonen kann ebenfalls eine Gewichtung oder Einschränkung vorgenommen werden, wenn die Software lediglich für eine bestimmte Benutzergruppe bestimmt ist.\\
\\
Neben Metriken sind vor allem Aufmerksamkeits- und Aktivitätsverteilungen eine wichtige Methode zur Analyse der Usability. Diese stammen ursprünglich aus dem Eye-Tracking Verfahren und dienen dort vor allem zum Test der Wirksamkeit verschiedener Objekte und Elemente einer Software. Das Ergebnis der Verteilung kann auf verschiedenster Art und Weise visuell dargestellt werden. Zur Markierung der besonders aktiven Bereiche wird oft rot verwendet. Um so mehr die Aktivität abnimmt, um so kälter wird die Farbe. Zur Bestimmung der Aufmerksamkeit wird beim Eye-Tracking Verfahren die Bewegung der Pupille aufgezeichnet, um daraus den Focus des Auges möglichst genau zu bestimmen. Da dieses Verfahren recht aufwendig und teuer ist, kann im Bereich des Web-Usability-Testing auch auf die Mausposition zurückgegriffen werden. Da viele User die Maus benutzen, um Bereiche auf einer Internetseite zu entdecken, erzeugt dieses Mittel ebenfalls ein aussagekräftes Ergebnis.


\subsection{Clickmaps}

Die Aufmerksamkeit oder Aktivität eines Users kann durch verschiedene Faktoren bestimmt werden. Bei der Clickmap konzentriert sich dies auf jeden Klick, die der User auf der Seite tätigt, sei es in der Navigation oder auf einen Link im Text. Als ein Overlay über den Screenshot einer Seite zeigt es ziemlich genau an, wohin der User geklickt hat. Dabei stellt sich oft heraus, dass Elemente angeklickt werden, die lediglich von grafischer Bedeutung sind. Benutzer missinterpretieren häufig Bilder als Links zu anderen Seiten. Durch Clickmaps kann dieses Problem sehr leicht erkannt werden.

\vspace{0.3cm}
\begin{center}
\includegraphics[scale=0.35]{./images/clickmap}
\end{center}
\begin{figure}[htb]
   \centering
   \caption{Beispiel einer Clickmap der \textit{thEvaluator} Projektwebsite}
    \label{clickmap}
\end{figure}


\subsection{Heatmap}

Bei Heatmaps bestimmt ein anderer Faktor die Aufmerksamkeit und Aktivität eines Benutzers. Hier wird neben dem Klickverhalten auch noch die $ x $ und $ y $ Position der Maus mit in die Berechnung der Heatmap mit einbezogen. Dadurch lassen sich große Datenwerte vereinfacht und anschaulich darstellen. Es ist teilweise sogar möglich die Bewegung der Maus genau mitzuverfolgen. In Abbildung \ref{heatmap} werden die gleichen Daten verwendet, die bereits für die Abbildung \ref{clickmap} benutzt wurden. Hier lassen sich eindeutige Mausbewegungen verfolgen. Der Focus der Aufmerksamkeit dieser Seite ist klar erkennbar.

\vspace{0.3cm}
\begin{center}
\includegraphics[scale=0.35]{./images/heatmap}
\end{center}
\begin{figure}[htb]
   \centering
   \caption{Beispiel einer Heatmap der \textit{thEvaluator} Projektwebsite}
    \label{heatmap}
\end{figure}


\subsection{Gazespots}

Aktivität kann auch durch Stillstand ausgedrückt werden. Betrachtet der Nutzer einen Punkt auf der Website intensiv, indem die Maus sich kaum dabei von der aktuellen Position fortbewegt, so kann man dies als Fixierung dieses Bereiches interpretieren und damit als Aufmerksamkeitsbereich. Je höher also die Betrachtungszeit eines Bereiches auf der Website ist, desto wärmer ist die Farbe auf dem Screenshot. Abbildung \ref{gazespot} zeigt, ebenfalls mit den gleichen Daten wie bei den letzten Abbildungen, die Fixierungspunkte der Projektwebsite.

\vspace{0.3cm}
\begin{center}
\includegraphics[scale=0.35]{./images/gazespot}
\end{center}
\begin{figure}[htb]
   \centering
   \caption{Beispiel einer Gazespot Visualisierung der \textit{thEvaluator} Projektwebsite}
    \label{gazespot}
\end{figure}


\subsection{Gazeplots}

Gazeplots konzentrieren sich ebenfalls auf Fixpunkte. Jedoch wird diesmal nicht durch Farbe ausgedrückt, wo die stärksten \Gls{Fixationen} auftreten, sondern durch nummerierte Kreise. Sie repräsentieren die Fixationsdauer. In den Kreisen steht eine numerische Zahl, die die Reihenfolge der Fixation angibt. Dadurch fließt eine zeitliche Dimension in die Auswertung mit ein. Linien verbinden letztendlich die Kreise und verdeutlichen somit den Bewegungsablauf der Maus. Dies wird auch als Sakkade bezeichnet.

\vspace{0.3cm}
\begin{center}
\includegraphics[scale=0.35]{./images/gazeplot}
\end{center}
\begin{figure}[htb]
   \centering
   \caption{Beispiel einer Gazeplot Visualisierung der \textit{thEvaluator} Projektwebsite}
    \label{gazeplot}
\end{figure}


\subsection{Attention-Maps}

Webseiten tendieren dazu, in der vertikalen Achse länger zu sein, als in der horizontalen. Dies liegt daran, dass das vertikale Scrollen viel natürlicher wirkt als das horizontale. Ein Nachteil haben trotzdem beide Arten. Die Wahrnehmung des Users sinkt, je weiter er scrollen muss. Wichtige Inhalte sollten daher immer direkt zu sehen sein. Dieser Usability Faktor kann ebenfalls durch eine Methode analysiert werden. Durch Attention-Maps (auch Scroll-Maps genannt) werden wieder über Farbverläufe Bereiche gekennzeichnet, die die meiste Aufmerksamkeit beim User erlangen. Inhalte, die durch den Bildschirm erfasst wurden, erhalten eine warme Farbe, hingegen Bereiche, zu denen der User gar nicht gekommen ist, da er nicht so weit gescrollt hat, eine kältere. Je nach Seitentyp ergibt dies meist ein einheitliches Muster. Die oberen Inhalte einer Seite werden immer rot gekennzeichnet. Je länger die Seite dann wird, desto kälter verläuft der Farbverlauf. Die Analyse über Attention Maps zeigt auf, ob eine Seite gekürzt werden muss oder nicht.\\
\\
Die Aufmerksamkeitsspanne beim Scrollen ist dabei stark abhängig vom Benutzertyp. Je nach Art der Seite verlieren User schneller oder langsamer die Lust daran. Bei Online-Shops stört dies beispielsweise nicht. Der Besucher stöbert hier gern durch lange Seiten und fühlt sich sogar beim Surfen gestört, wenn er schon nach kurzer Zeit eine neue Seite öffnen muss, um neue Produkte zu sehen. Oft findet man hier auch die Technik des \textit{Infinite Scrolling} wieder. Dabei werden Inhalte automatisch nachgeladen, sobald der User ans Ende der Seite gelangt. Dadurch wird ein Gefühl suggeriert, die Seite würde nie enden. Da der Besucher dabei die Inhalte schnell erforschen will, stört es ihn nicht, dass er dafür scrollen muss. Die Aufmerksamkeit bleibt nahezu konstant. Weitere Seiten, bei den dies der Fall ist, sind Social Networks, Blogs oder Portfolio Seiten.\\
\\
Ganz anders hingegen ist es bei informationsbasierten Seiten mit viel mehr Text als Bild. Hier sucht der Nutzer nach bestimmten Informationen und hat weder Zeit noch Lust lange auf einer Seite nach diesen zu suchen. Das Vertrauen in die Qualität der Inhalte nimmt umso mehr ab, je länger er dafür auf der Seite suchen oder scrollen muss.
%
% Marktanalyse
% Abschlussarbeit (Bachelor)
%
% Thema: Erstellung einer Browser Extension zur Usability Evaluierung von beliebigen Web-Applikationen über Heatmaps.
% Betreuer 1: Prof. Dr. Targo Pavlista
% Betreuer 2: Siamak Haschemi
%
% @author Christian Bromann <contact@christian-bromann.com>
%

\section{Marktanalyse}

% Analyse schon vorhandener Produkte
% zbs: loop11.com, met.picnet.com.au, crazyegg.com
%
% Mensch vs. Maschiene
% Abschlussarbeit (Bachelor)
%
% Thema: Erstellung einer Browser Extension zur Usability Evaluierung von beliebigen Web-Applikationen über Heatmaps.
% Betreuer 1: Prof. Dr. Targo Pavlista
% Betreuer 2: Siamak Haschemi
%
% @author Christian Bromann <contact@christian-bromann.com>
%

\section{Mensch vs. Maschine}

Zu Beginn dieses Kapitels wurde Usability als eine Art Bestreben nach ergonomischer Software definiert. In dem Mensch-Maschienen-System spielte dabei der User eine wichtige Schlüsselkomponente zum Erreichen dieses Vorhabens. Je nach Alter, Erfahrung und Intention nutzen Menschen eine Software sehr unterschiedlich. Dadurch scheint es, als ob immer eine Person zur Bestimmung der Usability involviert sein muss, um ein aussagekräftiges Resultat zu erzielen. Es stellt sich die Frage auf, ob dies wirklich so ist!\\
\\
Der Begriff \textit{Informatik} ist die Komposition aus den Wörtern Information und Automatik / Mathematik. Jeder Informatiker strebt danach, für doppelt ausgeführte Handlungen einen Automatismus zu entwerfen, um sich unnötige Arbeit zu sparen. Dies soll jedoch nicht suggerieren, dass Informatiker faule Menschen sind \cite{lazy}. Beim Usability-Engineering findet ebenfalls ein immer wieder auftretender Prozess der Analyse und Verbesserung statt. Dieser ist meist immer mit hohen Kosten und Zeitaufwänden verbunden. Zudem führen Usability-Tests aufgrund des sogenannten \textit{\glqq Evaluator Effect\grqq{}} \cite{anzahlTestpersonen} nicht immer zu den gewünschten Ergebnissen. Ein Grund dafür ist die kritische Rolle der Gutachter. Studien haben bewiesen, dass diese zu unterschiedlichen und teilweise falschen Interpretationen neigen und Probleme beweisen, die bereits entdeckt wurden.\\
\\
Nicht nur aufgrund dieser Tatsachen haben sich bereits viele Usabiliy Experten mit der Frage beschäftigt, wie die Benutzerfreundlichkeit auf Webseiten automatisiert getestet und in Ergebnisse überführt werden kann \cite{automatisierteUsabilityTests}. Daraus entstanden im Laufe der Zeit verschiedene Prototypen, die sich bereits auf dem Markt etablieren konnten. Diese nutzen entweder die bisher entwickelten Normen der Organisationen für Standardisierung oder versuchen ein mögliches Nutzerverhalten vorauszusehen. Letztere Herangehensweise wurde bereits von vielen Wissenschaftlern kritisiert, da die dafür verwendeten Algorithmen zum Vorhersagen des Verhaltens zu teilweise beirrenden Ergebnissen führen.\\
\\
Anders dagegen nutzt das Framework \textit{USEFul} ein Regelset, welches den verschiedenen Normen der Usability entspricht, zur Analyse der Seite und zur Einschätzung der Benutzerfreundlichkeit. Dieses Regelset wird dabei in drei verschiedene Kategorien untergliedert. Eine Kategorie entspricht den Richtlinien, die das Tool im vollem Umfang evaluieren und kontrollieren kann. Sie sind ablesbar und über Parameter klassifizierbar. Die nächste Kategorie beinhaltet Normen, die nur teilweise erfassbar und messbar sind. Zu abstrakte Usability-Regeln fallen schließlich in die letzte Kategorie. Diese sollten von den Evaluatoren manuell kontrolliert werden, ob sie für die aktuell getestete Seite zutreffen oder nicht. Zusätzlich werden die Regeln von 1 bis 5 priorisiert, um wichtigen Guidelines eine höhere Gewichtung zu geben.\\
\\
Laut Nielsen ist die URL ein Teil des User Interfaces und bestimmt somit ebenfalls die Usability einer Webseite \cite{urlasui}. Sie sollte so kurz und prägnant wie nötig gehalten werden und sich an der Seitenstruktur orientieren. Diese nicht ganz unwichtige Regel ist ein gutes Beispiel, um aufzuzeigen, wie USEFull eine Website automatisiert evaluieren kann. Nachdem sich das Tool die Source Dateien der Seite gezogen hat, sucht es nach allen Link Tags auf der Seite. Diese Tags beinhalten die Ziel URL im \textit{href} Attribut. Überschreitet die URL die Länge der vorher festgelegten maximalen Anzahl von Zeichen, so verstößt die Seite gegen diese Regel. Berechnet man nun das Verhältnis der korrekten gegenüber der zu langen URLs, so erhält man sogar eine Qualitätsrate, die zwischen verschiedenen Tests verglichen werden kann. Abbildung \ref{usefull} zeigt die letztendliche Gesamtauswertung des Frameworks. Die Ergebnisse werden innerhalb der jeweiligen Kategorien aufgelistet, nach Priorität sortiert und erklärt.

\vspace{0.3cm}
\begin{center}
\includegraphics[scale=1]{./images/usefull}
\end{center}
\begin{figure}[htb]
   \centering
   \caption{Ansicht einer automatisierten Auswertung des USEFull Frameworks\\\textbf{Quelle:} http://usabilitygeek.com/wp-content/uploads/2012/03/USEFul-A-Framework-To-Automate-Website-Usability-Evaluation-Analysis.jpg}
    \label{usefull}
\end{figure}

Unglücklicherweise ist der Prototyp noch nicht für die Öffentlichkeit zugänglich. Lediglich Papers und Usability Blogs beschreiben die Funktionsweise und Features des Tools. Nichtsdestotrotz scheint der Traum nach einer komplett automatisierten Usability Evaluierung auch damit nicht erfüllt zu sein. Da nicht wirklich alle Regeln mit Algorithmen überprüft werden können, bleibt es letztendlich die Aufgabe eines Experten zu entscheiden, ob die Ergebnisse der Auswertung repräsentativ genug sind, um die Qualität der Usability richtig einzustufen. Dennoch können viele Aspekte automatisiert kontrolliert werden und dadurch bei der Entwicklung frühzeitig die Probleme aufdecken, die ohne das Tool erst durch teure Tests erkannt werden.












