%
% Metriken
% Abschlussarbeit (Bachelor)
%
% Thema: Erstellung einer Browser Extension zur Usability Evaluierung von beliebigen Web-Applikationen über Heatmaps.
% Betreuer 1: Prof. Dr. Targo Pavlista
% Betreuer 2: Siamak Haschemi
%
% @author Christian Bromann <contact@christian-bromann.com>
%

\section{Metriken und Analyse}

Usability ist keine Eigenschaft, die einfach und exakt abzulesen ist. Um sie herauszufinden, gibt es zwei Möglichkeiten. Entweder man vertraut seinen eigenem Gefühl beim Benutzen der Software oder erstellt Tests und nutzte Metriken zur Auswertung der Ergebnisse. Die Intuition ist wichtig und kann viel aussagen, bietet jedoch keine Sicherheit, da sie lediglich die eigene Meinung repräsentiert. Da eine Software jedoch von verschiedenen Personen benutzt wird, die jeweils unterschiedliche Ansprüche an das Produkt stellen, reicht das eigene Gefühl nicht aus. Eine Usability Studie bringt ein allgemeineres Abbild der Benutzbarkeit. Die aus den Tests erzeugten Daten lassen sich durch verschiedenste Metriken analysieren und auswerten.\\
\\
Usability Tests unterliegen keinen Richtlinien. Sie sind sehr unterschiedlich und spezialisieren sich teilweise auf bestimmte Aspekte bei der Benutzung einer Software. Abhängig davon, welche Aspekte analysiert werden und welche Daten dabei aufgezeichnet werden, müssen Metriken ganze individuell für jeden Test erstellt werden. Sie bilden das Maß der Benutzbarkeit und sollten objektiv bestimmbar und vergleichbar sein.

\subsection{Clickmaps}
\subsection{Heatmaps}
\subsection{Gazespots}
\subsection{Gazeplots}
\subsection{Attention Maps}