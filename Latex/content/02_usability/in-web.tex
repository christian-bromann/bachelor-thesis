%
% Usability I'm Web
% Abschlussarbeit (Bachelor)
%
% Thema: Erstellung einer Browser Extension zur Usability Evaluierung von beliebigen Web-Applikationen über Heatmaps.
% Betreuer 1: Prof. Dr. Targo Pavlista
% Betreuer 2: Siamak Haschemi
%
% @author Christian Bromann <contact@christian-bromann.com>
%

\section{Usability im Web}

Das Thema Usability beschäftigt viele Bereiche der IT-Branche, doch nirgends erfährt sie so viel Aufmerksamkeit wie im Web. Hier ist eine hohe Benutzerfreundlichkeit essenziell, da der Besucher sofort die Webseite verlässt, wenn ihn etwas stört. Nirgendwo anders bestimmt das Wohlgefühl des Nutzers über die Dauer der Nutzung eines Produktes. Es ist der wichtigste Erfolgsfaktor eine Seite. Nicht nur deshalb beschäftigen sich so viele Menschen mit dem Thema Usability im Web. Es gibt mittlerweile dutzende Richtlinien und Standards, die verschiedene Regeln für Design, Struktur und Inhalt vorgeben. Einer der größten sind die \glqq Research-Based Web  Design \& Usability Guidelines\grqq{}, veröffentlicht vom U.S. Department of Health and Human Services, die ISO Norm 9241 sowie die Studie der JISC (JointInformation Systems Committee for higher education) über \glqq Guidelines For Academic Web Sites\grqq{} \cite{isoStandards}. Jedes Jahr tauschen sich Experten auf vielen Konferenzen über das Themas Usability aus und erhalten Preise von etlichen Usability Award Stiftungen. Dies zeigt wie groß der Focus auf diese Thematik liegt und unterstreicht ihre Wichtigkeit.\\
\\
Nach Jakob Nielsen bezieht sich Usability auf Eigenschaften wie Erlernbarkeit, Effizienz, Einprägsamkeit, Fehlerbehandlung und Nutzerzufriedenheit. Projiziert man dies auf das Web, so müssen die Definition der Attribute etwas verfeinert werden, um sie auf Webapplikationen anwenden zu können. Der Grund des Besuches einer Webseite basiert entweder auf das Suchen von bestimmten Informationen oder die Nutzung von Services, die dort angeboten werden. Die Applikation muss dabei das Vorhaben des Besuchers unterstützten, indem sie durch klare Trennung von Inhalt und Navigation leicht zu erlernen ist, ihre Inhalte effizient und schnell zugänglich macht und nach langer Abwesenheit immer noch nutzbar ist. Zudem sollte sie bei einem Fehler dem User Hilfe anbieten und ihm eine Art Wohlgefühl bei der Nutzung bereiten \cite{UsabilityPriciples}.\\
\\
Aus diesen Richtlinien lassen sich viele einzelne Regeln erstellen, die bei der Entwicklung Beachtung finden sollten. Diese können in vier grobe Bereiche eingeteilt werden: Zugänglichkeit, Identität, Navigation und Inhalt. Jeder dieser Regeln sind nicht zwingend verpflichtend und treffen auch nicht immer für jede Seite zu.


\subsection{Zugänglichkeit}

Abgesehen vom eigentlichen Inhalt und Aussehen ist die Zugänglichkeit der erste Faktor, der die Usability einer Seite bestimmt. Sie ist Maßstab dafür, wie gut der Benutzer auf die Inhalte zugreifen kann. Dies beginnt mit der Ladezeit einer Seite. Die meisten Menschen surfen heutzutage mit einer schnellen DSL Verbindung und sehen sie Seite schon innerhalb von wenigen Sekunden. Braucht eine Seite zu lange, um die Inhalte anzuzeigen, so verliert der User schnell die Lust zu warten und besucht eine andere Seite. Da Webseiten immer medialer ausgestattet werden, muss der Browser immer mehr Daten laden. Es ist deshalb wichtig, dass nicht alles auf einmal, sondern datenlastige Elemente nachzuladen. So bekommt der Besucher schon nach kurzer Zeit etwas zu sein und muss nicht Ewigkeiten auf einen weißen Bildschirm starren. Eine weitere Möglichkeit, die große Datenflut in den Griff zu bekommen, ist, Bilder und eingebundene Scripte und Styles zu komprimieren, um sie als ein komplettes Paket an den Browser zu schicken.\\
\\
Sind die Inhalte einmal geladen, so müssen sie auch gut zu erkennen sein. Ein gesunder Kontrast zwischen Hintergrund und Text sollte daher immer vorliegen. Des Weiteren ist eine ausreichende Schriftgröße wichtig, um die Lesbarkeit des Textes zu gewährleisten. Diese hängt stark davon ab, auf welche Zielgruppe sich die Seite konzentriert. Um sicher zu gehen ist es von Vorteil dem Besucher eine Möglichkeit zu geben, die Schriftgröße manuell zu ändern.\\
\\
Ein weiteres Hindernis sind Zusatzkomponenten, die eine Seite benötigt, um ihren Inhalt darstellen darstellen zu können. Das bekannteste Beispiel hierfür ist Flash. Obwohl die Mehrheit der Besucher dieses Zusatzprogramm installiert haben, gibt es dennoch viele User, die dies nicht haben. Gerade Leute, die mit einem Smartphone ins Internet gehen, müssen auf diese Inhalte verzichten. Dabei gibt es oft keinen Grund externe Programme in den Browser einzubinden. Der heutige Standard rund um HTML, JavaScript und CSS bietet alle Möglichkeiten, eine gleichwertige Applikation ohne die Erweiterung durch Flash oder Silverlight zu erstellen.\\
\\
Das Thema Internetzugang über das Smartphone ist ebenfalls eine wichtige Thematik. Immer mehr Menschen surfen im Internet während sie unterwegs sind oder bequem über das Tablett auf dem Sofa. Heutzutage sind noch nicht alle Webseiten dafür optimiert. Die Darstellung auf den deutliche kleineren Bildschirmen ist meist nicht für die Nutzung dafür geeignet. Mit einem \textit{Responsive Design} kann eine Seite speziell für eine kleinere Auflösung ausgeliefert werden und bietet auch den mobilen Nutzer eine komfortable und benutzerfreundliche Nutzung der Seite.


\subsection{Identität}


\subsection{Navigation}
\subsection{Inhalt}

% Grundsatz and regeln und normen erklären, die usability beeinflussen (prominter ort für logo etc)
% http://usabilitygeek.com/an-introduction-to-website-usability-testing/
% http://usabilitygeek.com/wp-content/uploads/2011/06/International-Journal-of-Human-Computer-Interaction-IJHCI-Volume-2-Issue-1.pdf