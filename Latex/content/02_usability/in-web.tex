%
% Usability I'm Web
% Abschlussarbeit (Bachelor)
%
% Thema: Erstellung einer Browser Extension zur Usability Evaluierung von beliebigen Web-Applikationen über Heatmaps.
% Betreuer 1: Prof. Dr. Targo Pavlista
% Betreuer 2: Siamak Haschemi
%
% @author Christian Bromann <contact@christian-bromann.com>
%

% glossary entries 
\newglossaryentry{Breadcrumb}{name=Breadcrumb, description={Information für den Besucher einer Seite, in welchen Navigationskontext er sich befindet (z.B. Hauptseite $\rightarrow$ Unterseite $\rightarrow$ Unterunterseite)}}

\section{Usability im Web}
Das Thema Usability beschäftigt viele Bereiche der IT-Branche, doch nirgends erfährt sie so viel Aufmerksamkeit wie im Web. Hier ist eine hohe Benutzerfreundlichkeit essenziell, da der Besucher sofort die Webseite verlässt, wenn ihn etwas stört. Nirgendwo anders bestimmt das Wohlgefühl des Nutzers über die Dauer der Nutzung eines Produktes. Es ist der wichtigste Erfolgsfaktor einer Seite. Nicht nur deshalb beschäftigen sich so viele Menschen mit dem Thema Usability im Web. Es gibt mittlerweile dutzende Richtlinien und Standards, die verschiedene Regeln für Design, Struktur und Inhalt vorgeben. Einer der Größten sind die \glqq Research-Based Web  Design \& Usability Guidelines\grqq{}, veröffentlicht vom U.S. Department of Health and Human Services, die ISO Norm 9241 sowie die Studie der JISC (Joint Information Systems Committee) über \glqq Guidelines For Academic Web Sites\grqq{} \cite{isoStandards}. Jedes Jahr tauschen sich Experten auf vielen Konferenzen über das Thema Usability aus und erhalten Preise von etlichen Usability Award Stiftungen. Dies zeigt, wie groß der Fokus auf diese Thematik ist und unterstreicht ihre Wichtigkeit.\\
\\
Nach Jakob Nielsen bezieht sich Usability auf Eigenschaften wie Erlernbarkeit, Effizienz, Einprägsamkeit, Fehlerbehandlung und Nutzerzufriedenheit. Projiziert man dies auf das Web, so muss die Definition der Attribute etwas verfeinert werden, um sie auf Webapplikationen anwenden zu können. Der Grund des Besuches einer Webseite basiert entweder auf dem Suchen von bestimmten Informationen oder der Nutzung von Services, die dort angeboten werden. Die Applikation muss dabei das Vorhaben des Besuchers unterstützten, indem sie durch klare Trennung von Inhalt und Navigation leicht zu erlernen ist, ihre Inhalte effizient und schnell zugänglich macht und nach langer Abwesenheit immer noch nutzbar ist. Zudem sollte sie bei einem Fehler dem User Hilfe anbieten und ihm eine Art Wohlgefühl bei der Nutzung bereiten \cite{UsabilityPriciples}.\\
\\
Aus diesen Richtlinien lassen sich viele einzelne Regeln erstellen, die bei der Entwicklung Beachtung finden sollten. Diese können in vier grobe Bereiche eingeteilt werden: Zugänglichkeit, Identität, Navigation und Inhalt. Jede dieser Regeln ist nicht zwingend verpflichtend und trifft auch nicht immer für jede Seite zu.


\subsection{Zugänglichkeit}

\hyphenation{me-di-al-er}

Abgesehen vom eigentlichen Inhalt und Aussehen ist die Zugänglichkeit der erste Faktor, der die Usability einer Seite bestimmt. Sie ist Maßstab dafür, wie gut der Benutzer auf die Inhalte zugreifen kann. Dies beginnt mit der Ladezeit einer Seite. Die meisten Menschen surfen heutzutage mit einer schnellen DSL Verbindung und sehen die Seite schon innerhalb von wenigen Sekunden. Braucht diese jedoch zu lange, um die Inhalte anzuzeigen, so verliert der User schnell die Lust zu warten und besucht eine andere Seite. Da Webseiten immer medialer ausgestattet werden, benötigt der Browser immer mehr Daten vom Server. Es ist daher wichtig, dass beim Aufruf der Seite lediglich die wichtigsten Elemente geladen werden, um die initiale Ladezeit möglichst gering zu halten und den Nutzer schnellstmöglich mit Inhalt zu versorgen. Erst wenn dieser auf z.B. hochauflösende Bilder oder Videos zugreift, sollten diese nachgeladen werden. Eine weitere Möglichkeit, die große Datenflut in den Griff zu bekommen, ist, Bilder, eingebundene Scripts und Styles zu komprimieren, um sie als ein komplettes Paket an den Browser zu schicken.\\
\\
Sind die Inhalte einmal geladen, so müssen sie auch gut zu erkennen sein. Ein gesunder Kontrast zwischen Hintergrund und Text sollte daher immer vorliegen. Des Weiteren ist eine ausreichende Schriftgröße wichtig, um die Lesbarkeit des Textes zu gewährleisten. Diese hängt stark davon ab, auf welche Zielgruppe sich die Seite konzentriert. Um hier auf der sicheren Seite zu sein, ist es von Vorteil, dem Besucher eine Möglichkeit zu geben, die Schriftgröße manuell zu ändern.\\
\\
Ein weiteres Hindernis sind Zusatzkomponenten, die eine Seite benötigt, um ihren Inhalt darstellen zu können. Das bekannteste Beispiel hierfür ist Flash. Obwohl die Mehrheit der Besucher dieses Zusatzprogramm installiert haben, gibt es dennoch viele, bei denen dies nicht zutrifft. Gerade User, die mit einem Smartphone ins Internet gehen, müssen auf diese Inhalte meist verzichten, da das Device Flash nicht unterstützt oder den Inhalt aufgrund der Auflösung nicht geeignet darstellen kann. Dabei gibt es oft keinen Grund, externe Programme in den Browser einzubinden. Der heutige Standard rund um HTML, JavaScript und CSS bietet alle Möglichkeiten, eine gleichwertige Applikation ohne die Erweiterung durch Flash oder Silverlight zu erstellen.\\
\\
Immer mehr Menschen surfen mit ihrem Handy im Internet, während sie unterwegs sind, oder nutzen Tablets bequem auf dem Sofa. Heutzutage sind noch nicht alle Webseiten dafür optimiert. Die Darstellung dieser auf den deutlich kleineren Bildschirmen ist meist nicht für die Nutzung dafür geeignet. Mit einem \textit{Responsive Design} kann eine Seite speziell für eine kleinere Auflösung ausgeliefert werden und bietet auch dem mobilen Nutzer eine komfortable und benutzerfreundliche Nutzung.


\subsection{Identität}

Das Erste, was dem User durch den Kopf schießt sobald er eine Webseite aufruft, ist: \textit{\glqq Wer oder Was stellt sich hier vor?\grqq{}} oder \textit{\glqq Was wird hier geboten?\grqq{}}. Um relativ schnell das Vertrauen des Benutzers zu gewinnen, sollte das Corperate Design schnell erkennbar sein, bspw. durch eine prominente Position des Logos in der oberen Hälfte der Seite. Für eine schnelle Übersicht ist zudem eine kurze und klare Beschreibung über den Inhalt und den Service direkt beim Start von Vorteil.\\
\\
Eine Kontaktseite sollte ebenfalls immer direkt erreichbar sein. Gerade bei Seiten, die einen Service anbieten, ist dies enorm wichtig. Kommt es zu Fragen, da zbs. irgendetwas nicht richtig funktioniert, suchen Besucher immer die Möglichkeit der Kontaktaufnahme und sind enttäuscht, wenn sie dies nicht finden.


\subsection{Navigation}

Sobald der Besucher erst einmal weiß, von wem er die Seite besucht und was ihn hier erwartet, braucht er eine Navigation, die ihn an sein Ziel führt. Diese sollte konsistent und klar zu erkennen sein. Eine Hauptnavigation, die auf jeder Seite an der gleichen Position zu finden ist, führt den Besucher am einfachsten durch die Struktur der Seite.\\
\\
Ist diese Struktur jedoch sehr komplex, führt sie oft zu Orientierungsverlust und überfordert den Besucher. Dies ist häufig bei akademischen Webseiten der Fall. Hier werden verschiedenste Inhalte für ganz unterschiedliche Benutzergruppen angeboten. Diese haben es schwer, die gewünschten Informationen zu finden, da der Content tief verschachtelt in diversen Oberkategorien steckt. Die Erstellung einer geeigneten Link Struktur ist deshalb eine der wichtigsten und gleichzeitig schwersten konzeptionellen Aufgaben.\\
\\
\Gls{Breadcrumb}s können dafür eine erste Abhilfe bieten. Sie geben dem Besucher Auskunft darüber, wo sie sich auf der Seite befinden und geben die Möglichkeit, im Kontext der Navigation zurückzugehen. Eine weitere Methode, den Überblick über die Inhalte einer Webseite zu bewahren, ist, eine relativ flache Navigationshierarchie/-struktur aufzubauen. Ein Navigationsmenü sollte daher nie mehr als 3 Unterebenen besitzen. Jede Ebene ist zudem in einem eigenen Bereich zu halten, um die Kategorisierung möglichst gut abzustufen. Gerade bei Webseiten mit viel Inhalt hilft letztendlich eine interne Suche dem Besucher dabei, die gewünschten und meist sehr spezifischen Informationen zu finden.

\subsection{Inhalt}

Ein oft zitiertes Sprichwort besagt \glqq Content is King\grqq{}. Dies unterstreicht die Bedeutung des Inhalts einer Webseite in Anbetracht der Qualität und Aufbereitung. Es ist deshalb ebenfalls ein bedeutender Faktor, wenn es darum geht, eine benutzerfreundliche Internetseite ins World Wide Web zu stellen.\\
\\
Schon lange ist bekannt, dass das Lesen auf Computer Monitoren für den Menschen deutlich langsamer und ungenauer ist als auf normalen Papier \cite{screenvspaper}. Aus Usability-Sicht ist es daher wichtig, dass der Text einer ordentlichen Formatierung unterliegt. Dies beinhaltet die Wahl einer geeignet großen Schriftgröße, die konsistente Nutzung von Farbe und Textstilen sowie die eindeutige Unterteilung des Textes in Überschriften und Textblöcken. Lange Texte in langen Zeilen sind sehr mühsam zu lesen. Sätze sollten deshalb immer prägnant und kurz formuliert sein. Hier zählt: weniger ist mehr! Ebenfalls ist bekannt, dass die Mehrheit der Menschen die Texte lediglich überfliegen, anstatt sie Wort für Wort zu lesen \cite{howtoreadinweb}. Viele suchen dabei nach bestimmten Keywords, die sie interessieren. Dies wurde in unzähligen Usability-Tests bestätigt. Durch das Markieren wichtiger Wörter im Text unterstützt man den Leser beim Finden dieser Keywords.\\
\\
Einer der ersten Prinzipien einer journalistischen Ausbildung ist das Prinzip der \glqq umgekehrten Pyramide\grqq{}. Dieses besagt, dass wichtiger Inhalt immer als Erstes in einem Text Platz finden muss. Danach schliesst sich Detail für Detail an. Dadurch kann der Leser an jeder beliebigen Stelle aufhören zu lesen und hat trotzdem das Wichtigste erfasst. Im Web erfährt dieses Prinzip ebenfalls große Bedeutung, gerade in Hinsicht auf Usability. Da der Monitorbildschirm selten das komplette vertikale Ausmaß einer Seite erfassen kann, lesen viele nur das, was sie ohne zu scrollen lesen können. Viele Attention Maps zeigen deutlich, dass die Benutzer selten bereit sind, weit zu scrollen, um Inhalte auf der unteren Hälfte der Seite zu erfassen.\\
\\
Als Letztes ist zu beachten, dass jede Seite einen geeigneten und einmaligen Seitentitel erhält und über eine kurze und gut beschreibende URL zu finden ist. Dies fördert nicht nur das Ranking auf Suchmaschienen, sondern ist ebenfalls für den User viel einprägsamer und wiederfindbar.