%
% Hauptdokument
% Abschlussarbeit (Bachelor)
%
% Thema: Erstellung einer Browser Extension zur Usability Evaluierung von beliebigen Web-Applikationen über Heatmaps.
% Betreuer 1: Prof. Dr. Targo Pavlista
% Betreuer 2: Siamak Haschemi
%
% @author Christian Bromann <contact@christian-bromann.com>
%

\documentclass[
	bibliography=totoc,
	listof=totoc,							% falls Index verwendet, ergänze "index=totoc" zu den Optionen 
	BCOR=5mm,							% Rand für Bindung: 5mm
	DIV=12]{scrbook}
\usepackage{bibgerm}       					% deutsche Literaturverzeichnisse
\usepackage[utf8]{inputenc}		       			% Umlaute im Text
\usepackage{graphicx} 						% Einfügen von Grafiken  - für PDF-Latex: .pdf und .png (.jpg möglich, sollte aber vermieden werden)
\usepackage{setspace}
\usepackage{fancyvrb, listings}          			% Für Code Ausgaben mit Highlighting
\usepackage{caption}
\usepackage[usenames]{color}

\definecolor{lightgray}{RGB}{230,230,230}
\definecolor{darkgrey}{RGB}{88,88,88}
\definecolor{purple}{RGB}{153,17,153}
\definecolor{darkgreen}{RGB}{3,120,65}
\lstdefinelanguage{JavaScript}{
  keywords={typeof, new, true, false, catch, function, return, null, catch, switch, var, if, in, while, do, else, case, break},
  keywordstyle=\color{blue}\bfseries,
  ndkeywords={class, export, boolean, throw, implements, import, this},
  ndkeywordstyle=\color{darkgrey}\bfseries,
  identifierstyle=\color{black},
  sensitive=false,
  comment=[l]{//},
  morecomment=[s]{/*}{*/},
  commentstyle=\color{darkgreen}\ttfamily,
  stringstyle=\color{red}\ttfamily,
  morestring=[b]',
  morestring=[b]",
  emph=[1]{prototype,methodName,createElement,append,require,writeFile,log,asynchronousFunction,series,then},
  emphstyle=[1]\color{purple},
}
\lstset{
   language=JavaScript,
   backgroundcolor=\color{lightgray},
   extendedchars=true,
   basicstyle=\footnotesize\ttfamily,
   showstringspaces=false,
   showspaces=false,
   numbers=left,
   numberstyle=\footnotesize,
   numbersep=9pt,
   tabsize=2,
   breaklines=true,
   showtabs=false,
   captionpos=b,
   frame=single
}

\usepackage{url}           						% URL's (z.B. in Literatur) schöner formatieren
\usepackage{hyperref} 						% sorgt für Hyperlinks in PDF-Dokumenten
\usepackage[toc]{glossaries}						% Glossary Paket einbinden
\makeglossaries
\graphicspath{./images/}	 					% Pfad zu den Bildern des Dokumentes
\linespread{1.3}

\begin{document}

% ---------------------------------------------------------------

\frontmatter 								% Titelblätter und Erklärung
    %
% Hauptdokument
% Praktikumsbericht zur Anerkennung der Arbeit bei der Spot-Media AG
% @author Christian Bromann <contact@christian-bromann.com>
%

\thispagestyle{empty}

\begin{center}

\vspace*{-1cm}

\includegraphics[width=0.3\textwidth]{./images/beuth_logo}

\vspace{1cm}

Fachbereich VI\\
\href{http://fb6.beuth-hochschule.de}{Informatik und Medien}

\vspace{1.3cm}

{\Large \textbf{Bachelorarbeit}}\\ 

\vspace{1cm}

{\Huge \textbf{Erstellung einer Browser Extension}}\\
\vspace*{3mm}
{\Huge \textbf{zur Usability-Evaluierung von}}\\
\vspace*{3mm}
{\Huge \textbf{beliebigen Webapplikationen über}}\\
\vspace*{3mm}
{\Huge \textbf{Heatmaps}}\\

\vspace{1.5cm}

{\Large \textbf{Christian Bromann}}\\ 
\textbf{Matr. Nummer 774261}

\vspace{1.5cm}

\parbox{1cm}{
\begin{large}
\begin{tabbing}
Betreuer: \hspace{1.5cm}\=Prof. Dr. Targo Pavlista\\[1mm]
Gutachter: \>Prof. Knabe\\[2mm]
Abgabetermin: \> 22. Juli 2013\\
\end{tabbing}
\end{large}}\\

\end{center}

\newpage
 					% Titelblatt mit Beuth Logo, Titel und Name
    \thispagestyle{empty}
    \cleardoublepage
    %
% Erklärung zur selbstständigen Arbeit
% Praktikumsbericht zur Anerkennung der Arbeit bei der Spot-Media AG
% @author Christian Bromann <contact@christian-bromann.com>
%

\documentclass[
	bibliography=totoc,
	listof=totoc,							% falls Index verwendet, ergänze "index=totoc" zu den Optionen 
	BCOR=5mm,							% Rand für Bindung: 5mm
	DIV=12]{scrbook}
\usepackage{bibgerm}       					% deutsche Literaturverzeichnisse
\usepackage[utf8]{inputenc}		       			% Umlaute im Text
\usepackage{setspace}

\pagestyle{empty}
\linespread{1.3}

\begin{document}

\vspace*{1cm}

\noindent
{\Huge \textbf{Ehrenwörtliche Erklärung}}\\
\\
Hiermit erkläre ich, Christian Bromann, geboren am 04.08.1989 in Salzwedel, ehrenwörtlich, dass ich meine Bachelorarbeit mit dem Titel:\\
\\
\textit{„Erstellung einer Browser Extension zur Usability-Evaluierung von beliebigen Webapplikationen über Heatmaps“}\\
\\
selbstständig und ohne fremde Hilfe angefertigt und keine anderen als in der Abhandlung angegebenen Hilfen benutzt habe.
Die Übernahme wörtlicher Zitate aus der Literatur sowie die Verwendung der Gedanken anderer Autoren an den entsprechenden Stellen habe ich innerhalb der Arbeit gekennzeichnet.\\
\\
Ich bin mir bewusst, dass eine falsche Erklärung rechtliche Folgen haben kann.

\vspace{2cm}

\noindent
Berlin, den \today

\vspace{3cm}

\hspace*{7cm}%
\dotfill\\
\hspace*{9.5cm}%
\textit{Christian Bromann}

\end{document}
				% Erklärung zur selbstständigen Arbeit
    \vspace*{2cm}

\begin{center}
    \textbf{Abstract}
\end{center}

\vspace*{1cm}

\noindent Das Internet hat sich zu einer der wichtigsten Werkzeuge zur Kommunikation des Menschen entwickelt und nimmt eine immer wichtigere Rolle in unserem Alltag ein. Umso wichtiger ist es, dass der Mensch dieses Werkzeug richtig zu benutzen weiß, um dessen volle Funktionalität effizient ausschöpfen zu können. Um dies zu Erreichen wurde im Zuge dieser Arbeit ein Usability-Testing-Tool mit dem Namen \textit{thEvaluator} entwickelt, mithilfe dessen das Nutzerverhalten auf beliebigen Webseiten untersucht werden kann, um daraus Probleme zu evaluieren und Lösungsmöglichkeiten zu entwickeln. Nach der Durchführung zweier Tests mit dem Programm, konnten einmal am Tool selbst und an einer Hochschulseite insgesamt fünf wichtige Erkenntnisse über die Benutzung ausgewertet werden. Durch den Einsatz von verschiedensten Techniken und Metriken ermöglicht der entwickelte Prototyp eine flexible Untersuchung der Usability und kann dadurch bei der Erstellung von benutzerfreundlichen Internetseiten helfen.
 					% Abstract
    \addcontentsline{toc}{chapter}{\numberline{}Abstract}
    \thispagestyle{empty}
    \begin{spacing}{1.48}
        \tableofcontents						% Inhaltsverzeichnis
    \end{spacing}
% ---------------------------------------------------------------

\mainmatter 								% die eigentliche Arbeit

    %
% Ergebnis
% Abschlussarbeit (Bachelor)
%
% Thema: Erstellung einer Browser Extension zur Usability Evaluierung von beliebigen Web-Applikationen über Heatmaps.
% Betreuer 1: Prof. Dr. Targo Pavlista
% Betreuer 2: Siamak Haschemi
%
% @author Christian Bromann <contact@christian-bromann.com>
%

\chapter{Ergebnis}

Das Ziel dieser Arbeit war die Entwicklung eines Usability-Tools zur Analyse des Nutzerverhaltens auf beliebigen Webseiten über Heatmaps. Die im Rahmen des Tools definierten Testcases sollten einen aufgabenbasierten Charakter besitzen, um Verhaltensmuster der Besucher genauer abbilden zu können. Nach der Entwicklung des Tools, sollten mit zwei Tests die Funktionalität des Tools auf die Probe gestellt und die Usability von Hochschulseiten überprüft werden.\\
\\
Mit \textit{thEvaluator} ist ein Tool geschaffen worden, mit dem präzise Usability-Evaluierungen in Echtzeit möglich sind. Die verschiedenen Auswertungs-Widgets bieten eine flexible Analyse des Nutzerverhaltens und die Verfolgung der Mauspositionen eines jeden Benutzers an. Durch den Einsatz einer Browser-Extension zur Aufzeichnung der Daten kann das Tool auf jeder beliebigen Seite eingesetzt werden. Es bedarf keine Code-Anpassungen. Obwohl es dabei auf den neuesten Web-Technologien aufbaut, ist die Portierung der Extension auf andere Browser problemlos umsetzbar. Zudem ermöglichen die aufgabenbasierten Testcases die Nachahmung bestimmter Nutzungsszenarien. Der Einsatz von NodeJS als API-Lösung und Socket-Streams zur Datenübermittlung macht das Tool sehr performant und flexibel erweiterbar. Der mit dieser Arbeit entwickelte Prototyp lässt sich durch weitere Browser-APIs ergänzen und stellt eine echte Konkurrenz gegenüber den teuren Produkten auf dem Markt dar.\\
\\
Während der Entwicklung des Tools stellte sich heraus, dass für eine genaue Untersuchung der Usability eine Heatmap nicht ausreicht. Durch die Ergänzung von Clickmaps, Gazespots und Gazeplots, einer Feedbackbox, sowie der zeitlich versetzten Rekonstruktion der Mausbewegung, wurden weitere Möglichkeiten zur Analyse des Userverhaltens ergänzt. Zudem ist durch die Implementierung eines Baum-Diagrammes die Untersuchung von Besucherpfaden möglich. Dadurch hat das Tool den Charakter eines universalen Usability-Analyse-Werkzeuges erhalten, welches sich nicht nur auf Heatmaps beschränkt.\\
\\
Die Usability Tests am Ende der Arbeit gaben erste Erkenntnisse über die Benutzerfreundlichkeit des Tools selber und zeigten verschiedene Usability-Problemstellen auf der Seite der Beuth Hochschule auf. Die Analyse der Daten ergab für den ersten Test, dass das Formular zur Erstellung eines Testcases für den Benutzer nicht selbsterklärend genug ist und durch Hinweistexte ergänzt werden sollte. Bei der Analyse der Hochschulseite stellte sich heraus, dass es vereinzelt Usability-Probleme gibt und die Probanden beim Finden von Informationen vermehrt die Such-Funktion genutzt haben. Dennoch fiel das Ergebnis des Tests positiv aus und bewies, dass Besucher von Hochschulseiten zwar lange zum Finden von bestimmten Informationen brauchen, jedoch die Suche häufig erfolgreich ausgeht.


% ---------------------------------------------------------------

\backmatter 								% ab hier keine Nummerierung mehr
    \listoffigures								% Abbildungsverzeichnis
    \lstlistoflistings							% Listing
    \bibliographystyle{alpha}
    \bibliography{./bibliography/bibliography}
    \printglossaries

\end{document}
