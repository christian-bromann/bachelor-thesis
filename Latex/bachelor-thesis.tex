%
% Hauptdokument
% Abschlussarbeit (Bachelor)
%
% Thema: Erstellung einer Browser Extension zur Usability Evaluierung von beliebigen Web-Applikationen �ber Heatmaps.
% Betreuer 1: Prof. Dr. Targo Pavlista
% Betreuer 2: Siamak Haschemi
%
% @author Christian Bromann <contact@christian-bromann.com>
%

\documentclass[
	bibliography=totoc,
	listof=totoc,					% falls Index verwendet, erg�nze "index=totoc" zu den Optionen 
	BCOR=5mm,					% Rand f�r Bindung: 5mm
	DIV=12]{scrbook}
\usepackage{bibgerm}       			% deutsche Literaturverzeichnisse
\usepackage[applemac]{inputenc} 		% Umlaute im Text
\usepackage{graphicx} 				% Einf�gen von Grafiken  - f�r PDF-Latex: .pdf und .png (.jpg m�glich, sollte aber vermieden werden)

\usepackage{fancyvrb, listings, color} 	% F�r Code Ausgaben mit Highlighting
\usepackage{caption}

\definecolor{grey}{RGB}{240,240,240}
\definecolor{darkgrey}{RGB}{88,88,88}
\definecolor{brown}{RGB}{237,201,175}
\definecolor{darkbrown}{RGB}{146,111,91}
\definecolor{darkgreen}{RGB}{0,100,50}
\definecolor{darkpurple}{RGB}{75,0,130}
\definecolor{php}{RGB}{192, 64, 0}
\definecolor{vars}{RGB}{71, 110, 155}
\definecolor{functions}{RGB}{156, 112, 63}

\lstset{
    fancyvrb=true,
    showstringspaces=false,
    tabsize=3,
    keywordstyle=\color{blue}\textbf,
    commentstyle=\color{darkgreen}\textit,
    stringstyle=\color{red},
    backgroundcolor=\color{grey},
    emph=[1]{php},
    emphstyle=[1]\color{php},
    emph=[2]{public,function,use,as,class,private,namespace,return},
    emphstyle=[2]\color{darkpurple},
    emph=[3]{Signee},
    emphstyle=[3]\color{darkbrown},
    emph=[4]{null,array},
    emphstyle=[4]\color{brown},
    emph=[5]{$form,$request,$em,$this,$id,$createdAt,$name,$signee,$signees},
    emphstyle=[5]\color{vars},
    emph=[6]{getSigneeForm,getRequest,getMethod,getData,getDoctrine,getEntityManager,persist,flush,getRepository,render,findAll},
    emphstyle=[6]\color{functions}
}

\DeclareCaptionFont{white}{\color{white}}
\DeclareCaptionFormat{listing}{\colorbox{darkgrey}{\parbox{\textwidth}{#1#2#3}}}
\captionsetup[lstlisting]{format=listing,labelfont=white,textfont=white}

\usepackage{url}           				% URL's (z.B. in Literatur) sch�ner formatieren
\usepackage{hyperref} 				% sorgt f�r Hyperlinks in PDF-Dokumenten

\graphicspath{{./images/}}				% Pfad zu den Bildern des Dokumentes

\begin{document}

% ---------------------------------------------------------------

\frontmatter 						% Titelbl�tter und Erkl�rung
    %
% Hauptdokument
% Praktikumsbericht zur Anerkennung der Arbeit bei der Spot-Media AG
% @author Christian Bromann <contact@christian-bromann.com>
%

\thispagestyle{empty}

\begin{center}

\vspace*{-1cm}

\includegraphics[width=0.3\textwidth]{./images/beuth_logo}

\vspace{1cm}

Fachbereich VI\\
\href{http://fb6.beuth-hochschule.de}{Informatik und Medien}

\vspace{1.3cm}

{\Large \textbf{Bachelorarbeit}}\\ 

\vspace{1cm}

{\Huge \textbf{Erstellung einer Browser Extension}}\\
\vspace*{3mm}
{\Huge \textbf{zur Usability-Evaluierung von}}\\
\vspace*{3mm}
{\Huge \textbf{beliebigen Webapplikationen über}}\\
\vspace*{3mm}
{\Huge \textbf{Heatmaps}}\\

\vspace{1.5cm}

{\Large \textbf{Christian Bromann}}\\ 
\textbf{Matr. Nummer 774261}

\vspace{1.5cm}

\parbox{1cm}{
\begin{large}
\begin{tabbing}
Betreuer: \hspace{1.5cm}\=Prof. Dr. Targo Pavlista\\[1mm]
Gutachter: \>Prof. Knabe\\[2mm]
Abgabetermin: \> 22. Juli 2013\\
\end{tabbing}
\end{large}}\\

\end{center}

\newpage
 				% Titelblatt mit Beuth Logo, Titel und Name
    \thispagestyle{empty}
    \cleardoublepage
    %
% Erklärung zur selbstständigen Arbeit
% Praktikumsbericht zur Anerkennung der Arbeit bei der Spot-Media AG
% @author Christian Bromann <contact@christian-bromann.com>
%

\documentclass[
	bibliography=totoc,
	listof=totoc,							% falls Index verwendet, ergänze "index=totoc" zu den Optionen 
	BCOR=5mm,							% Rand für Bindung: 5mm
	DIV=12]{scrbook}
\usepackage{bibgerm}       					% deutsche Literaturverzeichnisse
\usepackage[utf8]{inputenc}		       			% Umlaute im Text
\usepackage{setspace}

\pagestyle{empty}
\linespread{1.3}

\begin{document}

\vspace*{1cm}

\noindent
{\Huge \textbf{Ehrenwörtliche Erklärung}}\\
\\
Hiermit erkläre ich, Christian Bromann, geboren am 04.08.1989 in Salzwedel, ehrenwörtlich, dass ich meine Bachelorarbeit mit dem Titel:\\
\\
\textit{„Erstellung einer Browser Extension zur Usability-Evaluierung von beliebigen Webapplikationen über Heatmaps“}\\
\\
selbstständig und ohne fremde Hilfe angefertigt und keine anderen als in der Abhandlung angegebenen Hilfen benutzt habe.
Die Übernahme wörtlicher Zitate aus der Literatur sowie die Verwendung der Gedanken anderer Autoren an den entsprechenden Stellen habe ich innerhalb der Arbeit gekennzeichnet.\\
\\
Ich bin mir bewusst, dass eine falsche Erklärung rechtliche Folgen haben kann.

\vspace{2cm}

\noindent
Berlin, den \today

\vspace{3cm}

\hspace*{7cm}%
\dotfill\\
\hspace*{9.5cm}%
\textit{Christian Bromann}

\end{document}
		% Erkl�rung zur selbstst�ndigen Arbeit
    \vspace*{2cm}

\begin{center}
    \textbf{Abstract}
\end{center}

\vspace*{1cm}

\noindent Das Internet hat sich zu einer der wichtigsten Werkzeuge zur Kommunikation des Menschen entwickelt und nimmt eine immer wichtigere Rolle in unserem Alltag ein. Umso wichtiger ist es, dass der Mensch dieses Werkzeug richtig zu benutzen weiß, um dessen volle Funktionalität effizient ausschöpfen zu können. Um dies zu Erreichen wurde im Zuge dieser Arbeit ein Usability-Testing-Tool mit dem Namen \textit{thEvaluator} entwickelt, mithilfe dessen das Nutzerverhalten auf beliebigen Webseiten untersucht werden kann, um daraus Probleme zu evaluieren und Lösungsmöglichkeiten zu entwickeln. Nach der Durchführung zweier Tests mit dem Programm, konnten einmal am Tool selbst und an einer Hochschulseite insgesamt fünf wichtige Erkenntnisse über die Benutzung ausgewertet werden. Durch den Einsatz von verschiedensten Techniken und Metriken ermöglicht der entwickelte Prototyp eine flexible Untersuchung der Usability und kann dadurch bei der Erstellung von benutzerfreundlichen Internetseiten helfen.
 			% Abstract
    \thispagestyle{empty}
    \tableofcontents 					% Inhaltsverzeichnis

% ---------------------------------------------------------------

\mainmatter 						% die eigentliche Arbeit

    \chapter*{Abstract}%
\addcontentsline{toc}{chapter}{\numberline{}Abstract}%

\chapter{Einleitung}
	\section{Motivation und Ausgangssituation}
	\section{Projektziel}
	\section{Aufbau der Arbeit}

\chapter{Usability}
% Was ist Usability
% http://www.webpagecontent.com/arc_archive/124/5/
	\section{Usability im Web}
	% Usability l�sst sich nicht via Testscript Testen, sondern h�ngt immer ab von den verschiedenen Einfl�ssen der Benutzer
	% http://usabilitygeek.com/an-introduction-to-website-usability-testing/
	% http://usabilitygeek.com/wp-content/uploads/2011/06/International-Journal-of-Human-Computer-Interaction-IJHCI-Volume-2-Issue-1.pdf
	\section{Einfl�sse}
	% Was kann die Benutzbarkeit beeinflussen
	% Barrierefreiheit
	% X-Browser Gleichheit
	\section{Metriken}
		\subsection{Clickmaps}
		\subsection{Heatmaps}
		\subsection{Gazespots}
		\subsection{Gazeplots}
	\section{Marktanalyse}
	% Analyse schon vorhandener Produkte
	% zbs: loop11.com, met.picnet.com.au, crazyegg.com
	\section{Mensch vs. Maschiene}
	% in wie weit kann durch Testscripts die Usability getestet werden
	% webdriverjs
	\section{Business-Case}
	% m�gliche Testprozesse
	% automatische vs. gesteuert

\chapter{thEvaluator}
	\section{Zielbestimmungen}
	% Entwickler muss f�r bestimmte Sachen Events definieren
	% vs.
	% Tool kommt ohne Definierung von Events aus
	\section{Produktfunktionen}
	% Was ist m�glich, wie weit gehen die Auswertungen
	\section{Produkteinsatz}
	% Wie/Wann kann das Tool genutzt werden
	\section{Extension}
		\subsection{Aufbau}
		% Aus was besteht eine Chrome Extension
		% Manifest erkl�ren
		% Einbindung ohne Google Store
		\subsection{Architektur}
		% http://developer.chrome.com/extensions/overview.html
		\subsection{Nachrichten�bermittlung}
		% http://developer.chrome.com/extensions/messaging.html
		\subsection{Datenaufzeichnung}
		% Welche Daten werden wann aufgezeichnet
		% Screenshot Erstellung
		% Problematik der erh�hten Datenmenge
	\section{API}
		\subsection{Architektur}
		% express, welche Models gibt es
		% Welche Dependencies , Deploymentprozess
		\subsection{Asynchronit�t in nodeJS}
		% M�glichkeiten zur L�sung �ber closures, callbacks, Modulen wie async
		\subsection{REST Schnittstellen}
		% Auflistung mit erforderten Werten / R�ckgabewerten
		\subsection{Socket Streams}
		% Auflistung mit erforderten Werten / R�ckgabewerten
		\subsection{Persistierung}
		% SQL vs NoSQL
		% MongoDB / Mongoose
	\section{Webapplikation}
		\subsection{Aufbau}
		% Tooling, Bower, NPM
		% State of the art frontend entwicklung
		% Welche Dependencies , Deploymentprozess
		\subsection{Architektur}
		% Backbone, Models (die DB Modelle abbilden) Collections, Views
		% MV* Erkl�rung
		\subsection{TestCase Definition}
		% warum welche Attribute und Bedeutung (zbs. Aufl�sung, required)
		\subsection{Datenaufbereitung}
		% wann werden welche Daten geladen (lazy loading)
		% handling der Datenmenge
		\subsection{Datenevaluation}
		% welche Auswertungswidgets gibt es / Bedeutung
		% wann kann man m�gliche Resultate ableiten
		\subsection{Qualit�tsanforderungen}
	\section{Zukunftsaussichten}
	% Einsatz der Webcam f�r Eyetracking
	
\chapter{Ergebnis}
	\section{Scope-Verschiebung w�hrend der Entwicklung}
	% Heatmaps sind nicht mehr das Einzige, was es wert ist zu untersuchen
	% thEvaluator als universelles Usability Tool
	\section{Evaluierung der Hochschulseite}
	\section{Ergebnisse}
	\section{Schl�sse}




% ---------------------------------------------------------------

\backmatter 						% ab hier keine Nummerierung mehr
    \listoffigures						% Abbildungsverzeichnis
    \lstlistoflistings					% Listing
    %\include{./Literaturverzeichnis/literaturverzeichnis}

\end{document}
